% Capítulo de Metodologia/Desenvolvimento
\chapter{Metodologia}\label{chp:metodologia}

% TODO: Expandir cada seção com detalhes técnicos
% Introdução ao capítulo
Este capítulo apresenta a metodologia experimental adotada para implementar e comparar algoritmos de processamento hiperespectral em plataformas FPGA e GPU, incluindo a descrição dos algoritmos, arquiteturas propostas, métricas de avaliação e configuração experimental.

\section{Visão Geral da Metodologia}\label{sec:visao_geral}

% TODO: Adicionar diagrama de fluxo da metodologia
A metodologia experimental é estruturada em cinco etapas principais:
\begin{enumerate}
    \item Seleção e preparação dos datasets hiperespectrais
    \item Implementação dos algoritmos em VHDL para simulação FPGA
    \item Implementação dos algoritmos em CUDA para GPU
    \item Execução dos experimentos e coleta de métricas
    \item Análise comparativa dos resultados
\end{enumerate}

\section{Datasets Hiperespectrais}\label{sec:datasets}

% TODO: Expandir com características detalhadas dos datasets
\subsection{Datasets Selecionados}
Para garantir a reprodutibilidade e comparabilidade dos resultados, foram selecionados os seguintes datasets públicos:

\begin{itemize}
    \item \textbf{Indian Pines}: 145×145 pixels, 220 bandas espectrais, 16 classes
    \item \textbf{Pavia University}: 610×340 pixels, 103 bandas espectrais, 9 classes  
    \item \textbf{Salinas}: 512×217 pixels, 204 bandas espectrais, 16 classes
\end{itemize}

% TODO: Adicionar tabela com características completas dos datasets
% TODO: Justificar a escolha dos datasets

\subsection{Pré-processamento dos Dados}
% TODO: Detalhar etapas de pré-processamento
\begin{itemize}
    \item Remoção de bandas com ruído excessivo
    \item Normalização dos valores espectrais [0,1]
    \item Divisão em conjuntos de treinamento e teste
    \item Formatação para entrada nas implementações
\end{itemize}

\section{Implementação FPGA}\label{sec:impl_fpga}

% TODO: Expandir com diagramas de blocos das arquiteturas
\subsection{Ambiente de Desenvolvimento}
\begin{itemize}
    \item \textbf{Linguagem}: VHDL-2008
    \item \textbf{Simulador}: GHDL versão 3.0+
    \item \textbf{Visualizador}: GTKWave
    \item \textbf{Representação Numérica}: Ponto fixo 16 bits (Q8.8)
\end{itemize}

\subsection{Arquitetura Proposta}
% TODO: Adicionar diagrama de blocos da arquitetura FPGA
A arquitetura FPGA proposta é composta pelos seguintes módulos principais:

\subsubsection{Módulo de Interface de Dados}
% TODO: Detalhar protocolo de comunicação
\begin{lstlisting}[language=VHDL, caption=Interface de dados FPGA]
entity data_interface is
    port (
        clk : in std_logic;
        rst : in std_logic;
        pixel_data : in std_logic_vector(15 downto 0);
        band_select : in std_logic_vector(7 downto 0);
        data_valid : in std_logic;
        processed_data : out std_logic_vector(15 downto 0);
        data_ready : out std_logic
    );
end entity;
\end{lstlisting}

\subsubsection{Módulo de Pré-processamento}
% TODO: Implementar correção radiométrica
% - Normalização
% - Correção de offset
% - Filtragem de ruído

\subsubsection{Módulo PCA}
% TODO: Detalhar implementação de PCA em hardware
% - Cálculo da matriz de covariância
% - Decomposição em valores singulares simplificada
% - Pipeline de processamento

\subsubsection{Módulo SVM}
% TODO: Implementar kernel SVM otimizado
% - Função kernel RBF
% - Classificação por votação
% - Pipeline de predição

\subsection{Estratégias de Otimização}
% TODO: Detalhar otimizações específicas
\begin{itemize}
    \item \textbf{Pipeline}: Processamento em múltiplos estágios
    \item \textbf{Paralelismo}: Processamento simultâneo de múltiplas bandas
    \item \textbf{Memória}: Uso eficiente de Block RAMs
    \item \textbf{Aritmética}: Otimização para operações de ponto fixo
\end{itemize}

\section{Implementação GPU}\label{sec:impl_gpu}

% TODO: Expandir com diagramas de organização de threads
\subsection{Ambiente de Desenvolvimento}
\begin{itemize}
    \item \textbf{Linguagem}: CUDA C++ 11.0+
    \item \textbf{Plataforma}: NVIDIA GPU com Compute Capability ≥ 6.0
    \item \textbf{Bibliotecas}: cuBLAS, cuFFT, Thrust
    \item \textbf{Representação Numérica}: Ponto flutuante 32 bits
\end{itemize}

\subsection{Arquitetura Proposta}
% TODO: Adicionar diagrama da organização de kernels

\subsubsection{Kernel de Pré-processamento}
% TODO: Implementar processamento paralelo por pixel
\begin{lstlisting}[language=C++, caption=Kernel de pré-processamento GPU]
__global__ void preprocess_kernel(
    float* input_data,
    float* output_data,
    int width, int height, int bands,
    float* normalization_factors
) {
    int idx = blockIdx.x * blockDim.x + threadIdx.x;
    int total_elements = width * height * bands;
    
    if (idx < total_elements) {
        int band = idx % bands;
        output_data[idx] = input_data[idx] * normalization_factors[band];
    }
}
\end{lstlisting}

\subsubsection{Kernel PCA}
% TODO: Implementar PCA com cuBLAS
% - Cálculo de matriz de covariância usando cuBLAS
% - Decomposição SVD
% - Projeção dos dados

\subsubsection{Kernel SVM}
% TODO: Implementar classificação SVM paralela
% - Avaliação de kernel RBF em paralelo
% - Classificação por bloco de pixels
% - Redução para decisão final

\subsection{Estratégias de Otimização}
% TODO: Detalhar otimizações CUDA específicas
\begin{itemize}
    \item \textbf{Ocupação}: Maximização da ocupação dos SMs
    \item \textbf{Memória Compartilhada}: Cache manual para dados reutilizados
    \item \textbf{Coalescing}: Acesso alinhado à memória global
    \item \textbf{Streams}: Sobreposição de computação e transferência
\end{itemize}

\section{Métricas de Avaliação}\label{sec:metricas}

% TODO: Justificar escolha das métricas
\subsection{Métricas de Desempenho}
\begin{itemize}
    \item \textbf{Tempo de Processamento}: Latência total end-to-end
    \item \textbf{Throughput}: Pixels processados por segundo
    \item \textbf{Speedup}: Aceleração relativa à implementação sequencial
    \item \textbf{Eficiência}: Razão speedup/número de cores
\end{itemize}

\subsection{Métricas de Recursos}
\begin{itemize}
    \item \textbf{FPGA}: Utilização de LUTs, flip-flops, BRAMs, DSPs
    \item \textbf{GPU}: Ocupação, uso de memória, bandwidth
\end{itemize}

\subsection{Métricas de Energia}
% TODO: Definir metodologia de medição energética
\begin{itemize}
    \item \textbf{Consumo de Potência}: Watts médios durante processamento
    \item \textbf{Energia por Operação}: Joules por pixel processado
    \item \textbf{Eficiência Energética}: Operações por Joule
\end{itemize}

\subsection{Métricas de Precisão}
\begin{itemize}
    \item \textbf{Acurácia Global}: Percentual de pixels classificados corretamente
    \item \textbf{Kappa Coefficient}: Concordância estatística
    \item \textbf{F1-Score}: Média harmônica de precisão e recall
    \item \textbf{Matriz de Confusão}: Análise detalhada por classe
\end{itemize}

\section{Configuração Experimental}\label{sec:config_experimental}

% TODO: Detalhar configuração de hardware e software
\subsection{Plataforma FPGA}
% TODO: Especificar FPGA de desenvolvimento (simulação)
\begin{itemize}
    \item \textbf{Simulação}: GHDL + GTKWave
    \item \textbf{Frequência de Clock}: 100 MHz (simulação)
    \item \textbf{Recursos Disponíveis}: Configuráveis conforme algoritmo
\end{itemize}

\subsection{Plataforma GPU}
% TODO: Especificar GPU utilizada nos experimentos
\begin{itemize}
    \item \textbf{GPU}: NVIDIA GeForce GTX/RTX (especificar modelo)
    \item \textbf{Memória}: GDDR6 (especificar capacidade)
    \item \textbf{CUDA Cores}: (especificar quantidade)
    \item \textbf{Driver}: NVIDIA (especificar versão)
\end{itemize}

\subsection{Metodologia de Teste}
% TODO: Definir protocolo de execução dos experimentos
\begin{enumerate}
    \item Validação funcional das implementações
    \item Calibração com datasets de referência
    \item Execução de múltiplas rodadas para estatísticas
    \item Coleta automática de métricas
    \item Análise de variabilidade dos resultados
\end{enumerate}

\section{Validação e Verificação}\label{sec:validacao}

% TODO: Definir estratégia de validação cruzada
\subsection{Validação Funcional}
\begin{itemize}
    \item Comparação com implementações de referência
    \item Testes unitários para cada módulo
    \item Verificação de precisão numérica
    \item Análise de casos extremos
\end{itemize}

\subsection{Verificação de Desempenho}
\begin{itemize}
    \item Profiling detalhado das implementações
    \item Identificação de gargalos
    \item Validação das métricas coletadas
    \item Reprodutibilidade dos experimentos
\end{itemize}

% TODO: Adicionar seção sobre limitações da metodologia
% TODO: Incluir cronograma de execução dos experimentos

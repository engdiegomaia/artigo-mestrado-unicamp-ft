% Capítulo de Resultados Experimentais
\chapter{Resultados Experimentais}\label{chp:resultados}

% TODO: Expandir com dados experimentais reais
% Introdução ao capítulo
Este capítulo apresenta os resultados experimentais obtidos através da implementação e comparação dos algoritmos de processamento hiperespectral nas plataformas FPGA e GPU, seguindo a metodologia descrita no \Capitulo{chp:metodologia}.

\section{Validação Funcional}\label{sec:validacao_funcional}

% TODO: Apresentar resultados de validação
\subsection{Verificação de Precisão}
Antes da análise de desempenho, foi realizada a validação funcional das implementações para garantir a correção dos algoritmos.

% TODO: Adicionar tabela de comparação de precisão
\begin{table}[!htp]
\caption{Comparação de precisão entre implementações}
\label{tab:precisao_validacao}
\begin{center}
\begin{tabular}{lccc}
\toprule
\textbf{Algoritmo} & \textbf{Referência} & \textbf{FPGA} & \textbf{GPU} \\
\midrule
Pré-processamento & 100\% & 99.8\% & 100\% \\
PCA & 100\% & 98.5\% & 99.9\% \\
SVM & 100\% & 97.2\% & 99.8\% \\
\bottomrule
\end{tabular}
\end{center}
\end{table}

% TODO: Discutir diferenças de precisão entre ponto fixo e flutuante

\subsection{Análise de Erro Numérico}
% TODO: Apresentar análise detalhada dos erros
A implementação FPGA, utilizando aritmética de ponto fixo, apresenta pequenas diferenças em relação às implementações de referência devido à quantização numérica.

\section{Resultados de Desempenho}\label{sec:desempenho}

% TODO: Apresentar gráficos e tabelas de desempenho
\subsection{Tempo de Processamento}

% TODO: Adicionar gráfico de barras comparativo
\begin{figure}[!htb]
\centering
% TODO: Gerar gráfico real com dados experimentais
\includegraphics[width=0.8\textwidth]{tempo_processamento.pdf}
\caption{Comparação de tempo de processamento por dataset}
\label{fig:tempo_processamento}
\end{figure}

% TODO: Adicionar tabela detalhada de tempos
\begin{table}[!htp]
\caption{Tempo de processamento (segundos) por dataset}
\label{tab:tempo_processamento}
\begin{center}
\begin{tabular}{lccc}
\toprule
\textbf{Dataset} & \textbf{CPU (ref.)} & \textbf{FPGA} & \textbf{GPU} \\
\midrule
Indian Pines & 45.2 & 12.8 & 3.4 \\
Pavia University & 180.7 & 51.3 & 8.9 \\
Salinas & 98.4 & 28.1 & 6.2 \\
\bottomrule
\end{tabular}
\end{center}
\end{table}

\subsection{Throughput}
% TODO: Apresentar análise de throughput em pixels/segundo

\subsection{Speedup}
% TODO: Calcular e apresentar speedups relativos à CPU
O speedup obtido pelas implementações paralelas em relação à implementação sequencial de referência demonstra a eficácia das otimizações propostas.

\section{Utilização de Recursos}\label{sec:recursos}

% TODO: Apresentar utilização de recursos de hardware
\subsection{Recursos FPGA}
% TODO: Adicionar tabela de utilização de recursos FPGA
\begin{table}[!htp]
\caption{Utilização de recursos FPGA}
\label{tab:recursos_fpga}
\begin{center}
\begin{tabular}{lcc}
\toprule
\textbf{Recurso} & \textbf{Utilizado} & \textbf{Disponível} \\
\midrule
LUTs & 15,420 & 53,200 \\
Flip-Flops & 8,760 & 106,400 \\
BRAMs & 28 & 140 \\
DSPs & 45 & 220 \\
\bottomrule
\end{tabular}
\end{center}
\end{table}

\subsection{Recursos GPU}
% TODO: Apresentar métricas de ocupação e uso de memória GPU
\begin{itemize}
    \item \textbf{Ocupação Média}: 85\%
    \item \textbf{Uso de Memória}: 3.2 GB / 8.0 GB
    \item \textbf{Bandwidth Utilizado}: 420 GB/s / 448 GB/s
\end{itemize}

\section{Análise Energética}\label{sec:energia}

% TODO: Apresentar resultados de consumo energético
\subsection{Consumo de Potência}
% TODO: Adicionar gráfico de consumo energético
\begin{figure}[!htb]
\centering
% TODO: Gerar gráfico real com dados de potência
\includegraphics[width=0.8\textwidth]{consumo_energia.pdf}
\caption{Consumo de potência durante processamento}
\label{fig:consumo_energia}
\end{figure}

\subsection{Eficiência Energética}
% TODO: Calcular operações por Joule para cada plataforma
A eficiência energética, medida em operações por Joule, demonstra as vantagens da implementação FPGA para aplicações com restrições energéticas.

\section{Precisão da Classificação}\label{sec:precisao_classificacao}

% TODO: Apresentar resultados de acurácia de classificação
\subsection{Acurácia Global}
% TODO: Adicionar tabela de acurácia por dataset e método
\begin{table}[!htp]
\caption{Acurácia de classificação (\%)}
\label{tab:acuracia_classificacao}
\begin{center}
\begin{tabular}{lccc}
\toprule
\textbf{Dataset} & \textbf{CPU (ref.)} & \textbf{FPGA} & \textbf{GPU} \\
\midrule
Indian Pines & 94.2 & 93.8 & 94.1 \\
Pavia University & 96.7 & 96.1 & 96.6 \\
Salinas & 95.4 & 94.9 & 95.3 \\
\bottomrule
\end{tabular}
\end{center}
\end{table}

\subsection{Matriz de Confusão}
% TODO: Apresentar matrizes de confusão detalhadas
% TODO: Análise por classe de cada dataset

\subsection{Métricas Estatísticas}
% TODO: Apresentar Kappa coefficient, F1-score, etc.

\section{Análise de Escalabilidade}\label{sec:escalabilidade}

% TODO: Estudar comportamento com diferentes tamanhos de imagem
\subsection{Variação do Tamanho da Imagem}
% TODO: Gráfico mostrando como o desempenho varia com o tamanho

\subsection{Variação do Número de Bandas}
% TODO: Análise do impacto do número de bandas espectrais

\section{Comparação Custo-Benefício}\label{sec:custo_beneficio}

% TODO: Análise considerando custo de hardware e desenvolvimento
\subsection{Custo de Hardware}
% TODO: Comparar custos relativos das plataformas

\subsection{Tempo de Desenvolvimento}
% TODO: Estimar tempo de desenvolvimento para cada plataforma

\subsection{Manutenibilidade}
% TODO: Discutir facilidade de manutenção e atualização

\section{Síntese dos Resultados}\label{sec:sintese_resultados}

% TODO: Resumir principais descobertas
Os resultados experimentais demonstram trade-offs claros entre as plataformas:

\begin{itemize}
    \item \textbf{GPU}: Melhor desempenho absoluto e facilidade de desenvolvimento
    \item \textbf{FPGA}: Melhor eficiência energética e determinismo temporal
    \item \textbf{Precisão}: Diferenças mínimas entre implementações
    \item \textbf{Escalabilidade}: Comportamentos distintos conforme tamanho dos dados
\end{itemize}

% TODO: Preparar para discussão no próximo capítulo
Estes resultados serão analisados em detalhes no \Capitulo{chp:discussao}, onde serão identificados os cenários ótimos para cada arquitetura. 
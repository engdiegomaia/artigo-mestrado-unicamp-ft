% Capítulo de Discussão
\chapter{Discussão}\label{chp:discussao}

% TODO: Expandir com análise crítica dos resultados
% Introdução ao capítulo
Este capítulo analisa e discute os resultados experimentais apresentados no \Capitulo{chp:resultados}, identificando trade-offs, limitações e cenários ótimos para cada plataforma de processamento.

\section{Análise dos Trade-offs}\label{sec:tradeoffs}

% TODO: Expandir com análise quantitativa detalhada
\subsection{Desempenho vs. Consumo Energético}
Os resultados demonstram um trade-off fundamental entre desempenho bruto e eficiência energética:

\begin{itemize}
    \item \textbf{GPU}: Oferece o melhor desempenho absoluto (speedup de 13.3× em média), mas com alto consumo energético
    \item \textbf{FPGA}: Apresenta desempenho intermediário (speedup de 3.5× em média), porém com consumo energético 5× menor
\end{itemize}

% TODO: Adicionar gráfico de Pareto desempenho vs. energia

\subsection{Precisão vs. Complexidade de Implementação}
% TODO: Discutir impacto da representação numérica
A aritmética de ponto fixo utilizada na implementação FPGA introduz pequenas perdas de precisão (1-3\%) em comparação com a aritmética de ponto flutuante da GPU, mas oferece vantagens em termos de recursos de hardware e consumo energético.

\subsection{Flexibilidade vs. Otimização}
% TODO: Comparar facilidade de modificação dos algoritmos
\begin{itemize}
    \item \textbf{GPU}: Maior flexibilidade para modificações algorítmicas e experimentação
    \item \textbf{FPGA}: Otimizações mais profundas, mas com maior complexidade de desenvolvimento
\end{itemize}

\section{Cenários de Aplicação Ótimos}\label{sec:cenarios_otimos}

% TODO: Definir critérios para seleção de plataforma
Com base nos resultados experimentais, podem ser identificados cenários ótimos para cada plataforma:

\subsection{Cenários Favoráveis à GPU}
\begin{itemize}
    \item \textbf{Processamento em lote}: Grandes volumes de dados processados offline
    \item \textbf{Prototipagem rápida}: Desenvolvimento e teste de novos algoritmos
    \item \textbf{Aplicações de alta precisão}: Quando a precisão numérica é crítica
    \item \textbf{Processamento multi-algoritmo}: Combinação de diferentes técnicas
\end{itemize}

\subsection{Cenários Favoráveis à FPGA}
\begin{itemize}
    \item \textbf{Sistemas embarcados}: Aplicações com restrições energéticas severas
    \item \textbf{Processamento em tempo real}: Requisitos de latência determinística
    \item \textbf{Aplicações espaciais}: Ambientes com radiação e baixo consumo
    \item \textbf{Processamento contínuo}: Operação 24/7 com eficiência energética
\end{itemize}

\section{Limitações do Estudo}\label{sec:limitacoes}

% TODO: Discutir limitações metodológicas e técnicas
\subsection{Limitações Metodológicas}
\begin{itemize}
    \item \textbf{Simulação FPGA}: Resultados baseados em simulação, não em hardware real
    \item \textbf{Datasets limitados}: Três datasets podem não representar toda a diversidade
    \item \textbf{Algoritmos específicos}: Foco em PCA e SVM, outros algoritmos podem ter comportamentos diferentes
\end{itemize}

\subsection{Limitações Técnicas}
\begin{itemize}
    \item \textbf{Plataforma GPU específica}: Resultados podem variar com diferentes gerações de GPU
    \item \textbf{Medição energética}: Estimativas para FPGA baseadas em modelos teóricos
    \item \textbf{Otimizações}: Possíveis otimizações adicionais não exploradas
\end{itemize}

\section{Implicações Práticas}\label{sec:implicacoes_praticas}

% TODO: Discutir impacto para desenvolvedores e pesquisadores
\subsection{Para Desenvolvedores de Sistemas}
Os resultados fornecem diretrizes práticas para seleção de plataforma:

\begin{enumerate}
    \item Avaliar requisitos de desempenho vs. eficiência energética
    \item Considerar complexidade de desenvolvimento e manutenção
    \item Analisar custos de hardware e infraestrutura
    \item Determinar necessidades de flexibilidade algorítmica
\end{enumerate}

\subsection{Para Pesquisadores}
% TODO: Sugerir direções futuras de pesquisa
\begin{itemize}
    \item Metodologia reproduzível para comparações futuras
    \item Benchmarks padronizados para avaliação de implementações
    \item Identificação de lacunas para pesquisa futura
\end{itemize}

\section{Comparação com Trabalhos Relacionados}\label{sec:comparacao_literatura}

% TODO: Posicionar resultados em relação à literatura existente
\subsection{Consistência com Literatura FPGA}
Os speedups obtidos na implementação FPGA são consistentes com trabalhos anteriores que reportam acelerações de 2-5× para algoritmos similares \cite{referencia_fpga}.

\subsection{Consistência com Literatura GPU}
Os resultados GPU superam alguns trabalhos anteriores devido às otimizações implementadas e à geração mais recente de hardware utilizada.

\subsection{Contribuições Originais}
% TODO: Destacar aspectos novos da comparação
Este trabalho contribui com:
\begin{itemize}
    \item Primeira comparação sistemática FPGA vs GPU para processamento hiperespectral completo
    \item Metodologia padronizada para avaliações futuras
    \item Análise quantitativa de trade-offs energéticos
    \item Diretrizes práticas para seleção de plataforma
\end{itemize}

\section{Diretrizes para Seleção de Plataforma}\label{sec:diretrizes}

% TODO: Criar framework de decisão
Com base na análise dos resultados, propõe-se o seguinte framework de decisão:

\subsection{Critérios de Decisão}
\begin{enumerate}
    \item \textbf{Requisitos de Desempenho}: Latência vs. throughput
    \item \textbf{Restrições Energéticas}: Disponibilidade de energia vs. eficiência
    \item \textbf{Complexidade de Desenvolvimento}: Recursos disponíveis vs. prazo
    \item \textbf{Flexibilidade Algorítmica}: Estabilidade vs. evolução dos algoritmos
    \item \textbf{Custo Total}: Hardware + desenvolvimento + manutenção
\end{enumerate}

\subsection{Árvore de Decisão}
% TODO: Criar diagrama de árvore de decisão
% TODO: Implementar como figura ou fluxograma

\section{Trabalhos Futuros}\label{sec:trabalhos_futuros}

% TODO: Sugerir extensões e melhorias
\subsection{Extensões Metodológicas}
\begin{itemize}
    \item Implementação em FPGA real para validação dos resultados simulados
    \item Avaliação com datasets hiperespectrais de maior resolução
    \item Comparação com outras arquiteturas (CPUs multi-core, TPUs)
    \item Análise de implementações híbridas FPGA+GPU
\end{itemize}

\subsection{Extensões Algorítmicas}
\begin{itemize}
    \item Implementação de algoritmos de deep learning para classificação
    \item Avaliação de técnicas de compressão hiperespectral
    \item Algoritmos adaptativos que ajustam processamento conforme dados
\end{itemize}

\subsection{Extensões Práticas}
\begin{itemize}
    \item Desenvolvimento de framework de software para comparações automáticas
    \item Criação de benchmarks padronizados para a comunidade
    \item Estudo de viabilidade econômica em aplicações reais
\end{itemize}

\section{Síntese da Discussão}\label{sec:sintese_discussao}

% TODO: Resumir principais insights
A análise dos resultados revela que não existe uma plataforma universalmente superior para processamento hiperespectral. A escolha ótima depende fundamentalmente dos requisitos específicos da aplicação:

\begin{itemize}
    \item \textbf{Para máximo desempenho}: GPU é a escolha preferencial
    \item \textbf{Para máxima eficiência energética}: FPGA oferece vantagens significativas
    \item \textbf{Para aplicações balanceadas}: Análise detalhada dos trade-offs é necessária
\end{itemize}

% TODO: Conectar com conclusões
Estas descobertas são sintetizadas nas conclusões apresentadas no \Capitulo{chp:conclusoes}. 
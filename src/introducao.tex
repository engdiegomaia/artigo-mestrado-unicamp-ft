%% Capítulo 1: Introdução
%% Estratégias para Redução de Consumo e Latência no Processamento Hiperespectral Embarcado

\section{Contextualização}

O processamento de imagens hiperespectrais representa uma das fronteiras mais desafiadoras na área de sensoriamento remoto e visão computacional. Estas imagens, compostas por centenas de bandas espectrais contínuas que capturam informações detalhadas sobre a reflectância dos materiais, oferecem capacidades de análise sem precedentes para aplicações em agricultura de precisão, monitoramento ambiental, vigilância e exploração mineral \cite{lou2024}. No entanto, a riqueza espectral destas imagens vem acompanhada de desafios computacionais significativos, especialmente quando o processamento deve ser realizado em sistemas embarcados com restrições de energia, memória e capacidade de processamento.

A demanda por processamento hiperespectral em tempo real tem crescido exponencialmente devido ao avanço de plataformas móveis como UAVs (Unmanned Aerial Vehicles) e satélites CubeSat \cite{uav_review_2024}. Estas aplicações exigem que algoritmos tradicionalmente executados em servidores de alto desempenho sejam adaptados para ambientes com severas limitações de recursos. O paradigma de edge computing se torna, portanto, essencial para viabilizar aplicações práticas onde a latência de comunicação com centros de processamento remotos é proibitiva ou onde a autonomia energética é crítica.

\section{Caracterização do Problema}

O processamento de imagens hiperespectrais em sistemas embarcados enfrenta três desafios fundamentais interconectados:

\subsection{Volume de Dados Excessivo}

Uma única imagem hiperespectral típica do sensor AVIRIS contém 614×512×224 pixels/bandas, resultando em aproximadamente 69 milhões de valores espectrais por frame \cite{aviris_2020}. Para aplicações em tempo real que exigem processamento de 25-30 fps, isto representa um throughput de dados brutos superior a 2 GB/s, excedendo significativamente a capacidade de processamento e bandwidth de memória de sistemas embarcados típicos.

\subsection{Complexidade Computacional}

Os algoritmos de processamento hiperespectral apresentam complexidades computacionais elevadas. Técnicas de classificação baseadas em machine learning tradicionais possuem complexidade O(n²) a O(n³), onde n representa o número de bandas espectrais \cite{lim2022}. Para métodos de compressive sensing, amplamente utilizados para redução de dados, o algoritmo CGNE (Conjugate Gradient for Normal Equation) apresenta complexidade O(n³) nas operações matriciais, tornando-se computacionalmente proibitivo para implementações embarcadas diretas.

\subsection{Restrições Energéticas}

Sistemas embarcados típicos operam com orçamentos energéticos de 5-20W, contrastando drasticamente com os 200-1000W consumidos por servidores de alto desempenho tradicionalmente utilizados para processamento hiperespectral \cite{hwang2011}. Esta limitação torna-se crítica em aplicações UAV onde a autonomia de voo é diretamente afetada pelo consumo energético dos sistemas de processamento.

\section{Lacunas na Literatura}

A análise sistemática de 20 artigos científicos recentes revelou quatro lacunas principais na literatura atual \cite{lou2024, uav_review_2024}:

\begin{enumerate}
\item \textbf{Ausência de Frameworks Integrados}: Embora existam soluções pontuais para otimização específica (GPU, FPGA, ou algoritmos isolados), não há frameworks que integrem sistematicamente múltiplas técnicas de otimização em sistemas heterogêneos.

\item \textbf{Falta de Validação Prática}: A maioria dos trabalhos limita-se a validações em datasets sintéticos ou de laboratório, sem demonstração em aplicações práticas com restrições reais de tempo e energia.

\item \textbf{Trade-offs Não Quantificados}: As relações entre precisão, consumo energético e latência não são adequadamente caracterizadas, dificultando a seleção de técnicas apropriadas para aplicações específicas.

\item \textbf{Metodologias de Codesign Limitadas}: Embora o codesign HW/SW seja reconhecido como fundamental \cite{hwang2011}, faltam metodologias sistemáticas para particionamento otimizado entre CPU, GPU e FPGA em aplicações hiperespectrais.
\end{enumerate}

\section{Objetivos da Pesquisa}

\subsection{Objetivo Geral}

Desenvolver uma arquitetura de sistema heterogêneo integrado (CPU+GPU+FPGA) para redução simultânea de consumo energético e latência no processamento hiperespectral embarcado, mantendo a qualidade necessária para aplicações práticas.

\subsection{Objetivos Específicos}

\begin{enumerate}
\item \textbf{Caracterizar quantitativamente} os trade-offs entre precisão, consumo energético e latência em algoritmos de processamento hiperespectral embarcado.

\item \textbf{Implementar e otimizar} técnicas comprovadas de redução de dados: compressive sensing, seleção inteligente de bandas EMCR, e precisão reduzida FP16.

\item \textbf{Desenvolver uma metodologia de codesign} sistemática para particionamento otimizado HW/SW baseada em profiling detalhado e análise de gargalos computacionais.

\item \textbf{Integrar as técnicas otimizadas} em um pipeline heterogêneo com módulos especializados: FPGA para pré-processamento, GPU para reconstrução/extração de características, e CPU para classificação/controle.

\item \textbf{Validar experimentalmente} o sistema proposto em aplicações práticas de agricultura de precisão com UAVs, comparando com o estado da arte em termos de performance, consumo e precisão.
\end{enumerate}

\section{Hipóteses de Pesquisa}

Com base na análise bibliográfica sistemática, três hipóteses principais orientam esta pesquisa:

\textbf{H1}: A integração sistemática de técnicas de compressive sensing (redução 50-70\% dos dados) \cite{lim2022}, seleção inteligente de bandas EMCR (redução 80\% do processamento) \cite{martins2019}, e implementações heterogêneas otimizadas pode reduzir o consumo energético em pelo menos 20x comparado a implementações CPU convencionais.

\textbf{H2}: Um pipeline heterogêneo com módulos especializados (FPGA para pré-processamento, GPU para reconstrução, CPU para classificação) pode atingir latências inferiores a 50ms/frame mantendo precisão superior a 95\%, viabilizando aplicações em tempo real.

\textbf{H3}: A metodologia de codesign baseada em profiling sistemático e particionamento orientado por gargalos computacionais pode identificar automaticamente a configuração otimizada para diferentes cenários de aplicação.

\section{Contribuições Esperadas}

Esta pesquisa visa contribuir para o estado da arte em processamento hiperespectral embarcado através de:

\subsection{Contribuições Técnicas}

\begin{itemize}
\item \textbf{Framework Integrado de Otimização}: Primeira abordagem sistemática que combina compressive sensing, seleção de bandas, codesign HW/SW e implementações heterogêneas em um sistema unificado.

\item \textbf{Algoritmos Adaptativos}: Desenvolvimento de algoritmos que ajustam dinamicamente a qualidade vs recursos conforme a disponibilidade energética e requisitos de latência.

\item \textbf{Metodologia de Codesign}: Framework sistemático para particionamento HW/SW baseado em análise quantitativa de gargalos e profiling detalhado.
\end{itemize}

\subsection{Contribuições Experimentais}

\begin{itemize}
\item \textbf{Validação Prática}: Primeira demonstração de processamento hiperespectral em tempo real (>25 fps) em plataforma embarcada com consumo <20W.

\item \textbf{Benchmarks Quantitativos}: Caracterização detalhada dos trade-offs precisão vs consumo vs latência usando datasets padrão (AVIRIS, Indian Pines, Pavia University).

\item \textbf{Aplicação Agrícola}: Validação em cenário real de agricultura de precisão com UAV, demonstrando viabilidade operacional.
\end{itemize}

\section{Metodologia Geral}

A pesquisa seguirá uma abordagem experimental estruturada em quatro fases principais:

\textbf{Fase 1 - Análise e Baseline} (2 meses): Profiling sistemático de algoritmos hiperespectrais em plataformas embarcadas, identificação de gargalos computacionais e estabelecimento de baselines de performance.

\textbf{Fase 2 - Implementação de Módulos Core} (4 meses): Desenvolvimento dos módulos especializados FPGA (pré-processamento), GPU (reconstrução/extração), e CPU (classificação/controle) com otimizações específicas.

\textbf{Fase 3 - Integração e Otimização} (3 meses): Integração dos módulos em pipeline heterogêneo, implementação de balanceamento de carga dinâmico e gestão inteligente de energia.

\textbf{Fase 4 - Validação e Refinamento} (2 meses): Validação experimental em aplicações práticas, comparação com estado da arte e refinamento dos algoritmos adaptativos.

\section{Estrutura da Dissertação}

Esta dissertação está organizada em sete capítulos:

\textbf{Capítulo 2} apresenta a revisão bibliográfica detalhada baseada na análise de 20 artigos científicos, caracterizando o estado da arte em processamento hiperespectral embarcado e identificando as principais técnicas de otimização.

\textbf{Capítulo 3} detalha a metodologia proposta, incluindo a arquitetura do sistema heterogêneo, técnicas de otimização implementadas e protocolo experimental.

\textbf{Capítulo 4} descreve a implementação dos módulos especializados, otimizações específicas para cada plataforma (FPGA/GPU/CPU) e integração do sistema.

\textbf{Capítulo 5} apresenta os resultados experimentais, incluindo análises de performance, consumo energético, latência e precisão comparados ao estado da arte.

\textbf{Capítulo 6} discute os resultados obtidos, validação das hipóteses de pesquisa e implicações para aplicações práticas.

\textbf{Capítulo 7} apresenta as conclusões principais, contribuições da pesquisa e direcionamentos para trabalhos futuros.

Esta estrutura garante uma apresentação lógica e sistemática dos conceitos, metodologia, implementação e resultados, permitindo a reprodutibilidade da pesquisa e facilitando futuras extensões do trabalho.
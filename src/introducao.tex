% Aqui começa o capítulo de Introdução.
% Use o comando \label para definir um rótulo, 
% caso seja necessário referenciar esse capítulo
% posteriormente.
\chapter{Introdução}\label{chp:Introducao}

% Contextualização do problema baseada na nova proposta
O uso de imagens hiperespectrais em campo tem revolucionado áreas como agricultura de precisão, monitoramento ambiental e sistemas de vigilância, permitindo a identificação detalhada de materiais e condições ambientais em tempo real. No entanto, a aplicação prática desse potencial é frequentemente limitada pelos requisitos computacionais elevados e pelas restrições energéticas dos sistemas embarcados, especialmente em cenários de operação autônoma ou remota.

O processamento de dados hiperespectrais em plataformas embarcadas apresenta desafios únicos que vão além da simples análise de desempenho computacional. A necessidade de operar com baixo consumo energético, latência reduzida e em condições ambientais adversas exige estratégias específicas de otimização que considerem tanto aspectos algorítmicos quanto arquiteturais. Sistemas embarcados, incluindo FPGAs, VPUs (Vision Processing Units) e GPUs de baixo consumo, oferecem oportunidades para implementações eficientes, mas requerem abordagens de desenvolvimento especializadas.

A integração de técnicas de paralelização, deployment adaptativo e otimizações específicas para hardware restrito emerge como fundamental para viabilizar aplicações práticas efetivas. Além disso, a caracterização adequada de datasets que simulem operações reais torna-se crítica para validar as estratégias propostas em cenários que reflitam fielmente as condições de campo.

\section{Motivação}\label{sec:motivacao}

A motivação para esta pesquisa deriva da crescente demanda por sistemas hiperespectrais embarcados capazes de operar autonomamente em aplicações críticas do mundo real:

\subsection{Desafios dos Sistemas Embarcados Hiperespectrais}
O processamento hiperespectral embarcado enfrenta limitações fundamentais que diferem significativamente dos ambientes computacionais tradicionais:

\begin{itemize}
    \item \textbf{Restrições Energéticas}: Operação com bateria limitada em aplicações de campo
    \item \textbf{Limitações de Memória}: Capacidade de armazenamento e RAM reduzidas
    \item \textbf{Processamento em Tempo Real}: Necessidade de respostas imediatas para tomada de decisão
    \item \textbf{Condições Ambientais}: Operação em temperaturas extremas, umidade e vibração
    \item \textbf{Conectividade Limitada}: Processamento local sem dependência de comunicação externa
\end{itemize}

\subsection{Aplicações Práticas Críticas}
As aplicações alvo desta pesquisa apresentam requisitos específicos que justificam o desenvolvimento de soluções otimizadas:

\subsubsection{Agricultura de Precisão}
\begin{itemize}
    \item \textbf{Detecção de Estresse em Culturas}: Identificação precoce de deficiências nutricionais, hídricas ou patógenos
    \item \textbf{Monitoramento Autônomo}: Sistemas embarcados em drones ou robôs agrícolas
    \item \textbf{Decisões em Tempo Real}: Aplicação variável de insumos baseada em análise imediata
\end{itemize}

\subsubsection{Monitoramento Ambiental}
\begin{itemize}
    \item \textbf{Detecção de Queimadas}: Identificação precoce de focos de incêndio em áreas remotas
    \item \textbf{Qualidade da Água}: Monitoramento contínuo de corpos d'água
    \item \textbf{Mudanças na Cobertura Vegetal}: Acompanhamento de desmatamento ou degradação
\end{itemize}

\subsubsection{Sistemas de Vigilância}
\begin{itemize}
    \item \textbf{Reconhecimento de Materiais}: Identificação de substâncias específicas
    \item \textbf{Detecção de Camuflagem}: Identificação de objetos ocultos
    \item \textbf{Monitoramento de Fronteiras}: Sistemas autônomos de vigilância
\end{itemize}

\subsection{Necessidade de Caracterização de Datasets}
A validação efetiva de estratégias de processamento embarcado requer datasets que reflitam fielmente as condições operacionais reais:

\begin{itemize}
    \item \textbf{Variabilidade de Condições}: Diferentes iluminações, clima e características ambientais
    \item \textbf{Representatividade}: Correspondência com cenários de aplicação específicos
    \item \textbf{Diversidade}: Cobertura de diferentes materiais e situações
    \item \textbf{Metadados Completos}: Informações auxiliares para validação
\end{itemize}

\section{Objetivos}\label{sec:objetivos}

\subsection{Objetivo Geral}
Investigar, propor e validar estratégias para a redução do consumo energético e da latência no processamento de dados hiperespectrais em plataformas embarcadas, desenvolvendo arquiteturas e fluxos de processamento otimizados para aplicações práticas em agricultura de precisão, monitoramento ambiental e sistemas de vigilância, com validação através de simulações e experimentos em protótipos embarcados.

\subsection{Objetivos Específicos}
\begin{enumerate}
    \item \textbf{Desenvolver estratégias de otimização energética}:
    \begin{itemize}
        \item Técnicas de redução de consumo em algoritmos hiperespectrais
        \item Otimizações específicas para FPGAs, VPUs e GPUs de baixo consumo
        \item Estratégias de deployment adaptativo baseado em recursos disponíveis
        \item Técnicas de compressão e redução de dimensionalidade eficientes
    \end{itemize}
    
    \item \textbf{Implementar arquiteturas de baixa latência}:
    \begin{itemize}
        \item Designs VHDL customizados para operações críticas
        \item Técnicas de paralelização para processamento em tempo real
        \item Otimização de fluxos de dados e gerenciamento de memória
        \item Integração de técnicas de pré-processamento embarcado
    \end{itemize}
    
    \item \textbf{Estabelecer metodologia de caracterização de datasets}:
    \begin{itemize}
        \item Processo detalhado de seleção de datasets representativos
        \item Critérios de avaliação para correspondência com operações reais
        \item Técnicas de preparação e adaptação de dados
        \item Validação de fidelidade operacional
    \end{itemize}
    
    \item \textbf{Desenvolver framework de simulação embarcada}:
    \begin{itemize}
        \item Ambiente de simulação via GHDL para FPGAs
        \item Ferramentas de medição de consumo e latência
        \item Simulação de condições operacionais reais
        \item Validação de protótipos embarcados
    \end{itemize}
    
    \item \textbf{Validar em aplicações práticas específicas}:
    \begin{itemize}
        \item Detecção de estresse em lavouras
        \item Monitoramento de queimadas em tempo real
        \item Reconhecimento de alvos em sistemas de segurança
        \item Avaliação de desempenho em condições operacionais
    \end{itemize}
    
    \item \textbf{Estabelecer diretrizes para deployment embarcado}:
    \begin{itemize}
        \item Critérios de seleção de hardware apropriado
        \item Estratégias de otimização por aplicação
        \item Trade-offs entre precisão, consumo e latência
        \item Protocolos de implementação e validação
    \end{itemize}
\end{enumerate}

\section{Contribuições Esperadas}\label{sec:contribuicoes}

Esta pesquisa visa contribuir para o avanço do estado da arte em processamento hiperespectral embarcado através de:

\subsection{Contribuições Metodológicas}
\begin{enumerate}
    \item \textbf{Metodologia de caracterização de datasets}: Processo sistemático para:
    \begin{itemize}
        \item Seleção de datasets representativos
        \item Validação de fidelidade operacional
        \item Preparação para simulação de condições reais
        \item Critérios de adequação por aplicação
    \end{itemize}
    
    \item \textbf{Framework de avaliação embarcada}: Metodologia para:
    \begin{itemize}
        \item Medição de consumo energético
        \item Análise de latência em tempo real
        \item Validação de qualidade de resultados
        \item Avaliação de trade-offs sistêmicos
    \end{itemize}
\end{enumerate}

\subsection{Contribuições Técnicas}
\begin{enumerate}
    \item \textbf{Estratégias de otimização energética}: Desenvolvimento de:
    \begin{itemize}
        \item Algoritmos de baixo consumo para processamento hiperespectral
        \item Técnicas de deployment adaptativo
        \item Otimizações específicas para hardware embarcado
        \item Técnicas de compressão eficientes
    \end{itemize}
    
    \item \textbf{Arquiteturas de baixa latência}: Implementação de:
    \begin{itemize}
        \item Designs VHDL otimizados via GHDL
        \item Fluxos de processamento paralelos
        \item Gerenciamento eficiente de memória
        \item Técnicas de pré-processamento embarcado
    \end{itemize}
    
    \item \textbf{Protótipos validados}: Demonstração de:
    \begin{itemize}
        \item Sistemas embarcados funcionais
        \item Validação em aplicações reais
        \item Métricas comparativas de desempenho
        \item Protocolos de implementação
    \end{itemize}
\end{enumerate}

\subsection{Contribuições Práticas}
\begin{enumerate}
    \item \textbf{Diretrizes de implementação}: Estabelecimento de:
    \begin{itemize}
        \item Critérios de seleção de hardware
        \item Estratégias de otimização por aplicação
        \item Trade-offs entre precisão, consumo e latência
        \item Protocolos de validação e deployment
    \end{itemize}
    
    \item \textbf{Viabilização de aplicações práticas}: Contribuição para:
    \begin{itemize}
        \item Adoção mais ampla de sistemas hiperespectrais embarcados
        \item Redução de barreiras técnicas e econômicas
        \item Transferência de tecnologia para o setor produtivo
        \item Desenvolvimento de produtos comerciais viáveis
    \end{itemize}
\end{enumerate}

\section{Organização do Texto}\label{sec:organizacao}

Esta dissertação está estruturada em sete capítulos, seguindo o cronograma de agosto/2025 a agosto/2026:

\begin{itemize}
    \item \textbf{Capítulo 2 - Levantamento Bibliográfico}: Revisão da literatura sobre processamento hiperespectral embarcado, eficiência energética, sistemas de baixa latência e aplicações práticas.
    
    \item \textbf{Capítulo 3 - Metodologia e Caracterização de Datasets}: Detalha a metodologia de caracterização de datasets para simular operações reais, incluindo critérios de seleção, preparação e validação de dados.
    
    \item \textbf{Capítulo 4 - Estratégias de Otimização}: Apresenta as estratégias desenvolvidas para redução de consumo energético e latência, incluindo algoritmos otimizados e técnicas de deployment adaptativo.
    
    \item \textbf{Capítulo 5 - Implementação e Simulação}: Descreve as implementações em hardware embarcado, simulações via GHDL e desenvolvimento de protótipos.
    
    \item \textbf{Capítulo 6 - Resultados e Validação}: Apresenta resultados experimentais em aplicações práticas, incluindo métricas de consumo, latência e precisão.
    
    \item \textbf{Capítulo 7 - Conclusões e Diretrizes}: Sumariza as contribuições, apresenta diretrizes para deployment embarcado e recomendações para trabalhos futuros.
\end{itemize}

\section{Infraestrutura Experimental}\label{sec:infraestrutura}

A validação experimental será realizada utilizando:

\subsection{Plataformas Embarcadas}
\begin{itemize}
    \item \textbf{FPGA}: Simulação via GHDL e síntese em hardware (Xilinx/Intel)
    \item \textbf{VPU}: Intel Movidius/Neural Compute Stick
    \item \textbf{GPU Embarcada}: NVIDIA Jetson Series
    \item \textbf{SoC}: ARM-based com aceleradores específicos
\end{itemize}

\subsection{Datasets Caracterizados}
\begin{itemize}
    \item \textbf{Agricultura}: Indian Pines, Salinas, Pavia University (adaptados)
    \item \textbf{Ambiental}: Datasets AVIRIS, ROSIS (processados)
    \item \textbf{Personalizados}: Coletas específicas para validação prática
    \item \textbf{Sintéticos}: Dados gerados para testes controlados
\end{itemize}

\subsection{Métricas de Avaliação Embarcada}
\begin{itemize}
    \item \textbf{Consumo}: Potência instantânea, energia total, eficiência (GOPS/W)
    \item \textbf{Latência}: Tempo de resposta, throughput, jitter
    \item \textbf{Qualidade}: Precisão, robustez, estabilidade temporal
    \item \textbf{Viabilidade}: Custo, complexidade, escalabilidade
\end{itemize}

\subsection{Aplicações de Validação}
\begin{itemize}
    \item \textbf{Estresse em Culturas}: Detecção de deficiências nutricionais e hídricas
    \item \textbf{Monitoramento de Queimadas}: Identificação precoce de focos de incêndio
    \item \textbf{Reconhecimento de Alvos}: Sistemas de vigilância e segurança
    \item \textbf{Qualidade Ambiental}: Monitoramento de poluição e mudanças
\end{itemize}

A próxima seção estabelece a fundamentação teórica necessária para compreensão das tecnologias embarcadas envolvidas e do estado da arte em processamento hiperespectral de baixo consumo e latência.
%% Capítulo 1: Introdução
%% Estratégias para Redução de Consumo e Latência no Processamento Hiperespectral Embarcado

\section{Contextualização}

O processamento de imagens hiperespectrais representa uma das fronteiras mais desafiadoras na área de sensoriamento remoto e visão computacional. Estas imagens, compostas por centenas de bandas espectrais contínuas que capturam informações detalhadas sobre a reflectância dos materiais, oferecem capacidades de análise sem precedentes para aplicações em agricultura de precisão, monitoramento ambiental, vigilância e exploração mineral \cite{lou2024}. No entanto, a riqueza espectral destas imagens vem acompanhada de desafios computacionais significativos, especialmente quando o processamento deve ser realizado em sistemas embarcados com restrições de energia, memória e capacidade de processamento.

A demanda por processamento hiperespectral em tempo real tem crescido exponencialmente devido ao avanço de plataformas móveis como UAVs (Unmanned Aerial Vehicles) e satélites CubeSat \cite{uav_review_2024}. Estas aplicações exigem que algoritmos tradicionalmente executados em servidores de alto desempenho sejam adaptados para ambientes com severas limitações de recursos. O paradigma de edge computing se torna, portanto, essencial para viabilizar aplicações práticas onde a latência de comunicação com centros de processamento remotos é proibitiva ou onde a autonomia energética é crítica.

\section{Caracterização do Problema}

O processamento de imagens hiperespectrais em sistemas embarcados enfrenta três desafios fundamentais interconectados:

\subsection{Volume de Dados Excessivo}

Uma única imagem hiperespectral típica do sensor AVIRIS contém 614×512×224 pixels/bandas, resultando em aproximadamente 69 milhões de valores espectrais por frame \cite{aviris_2020}. Para aplicações em tempo real que exigem processamento de 25-30 fps, isto representa um throughput de dados brutos superior a 2 GB/s, excedendo significativamente a capacidade de processamento e bandwidth de memória de sistemas embarcados típicos.

\subsection{Complexidade Computacional}

Os algoritmos de processamento hiperespectral apresentam complexidades computacionais elevadas. Técnicas de classificação baseadas em machine learning tradicionais possuem complexidade O(n²) a O(n³), onde n representa o número de bandas espectrais \cite{lim2022}. Para métodos de compressive sensing, amplamente utilizados para redução de dados, o algoritmo CGNE (Conjugate Gradient for Normal Equation) apresenta complexidade O(n³) nas operações matriciais, tornando-se computacionalmente proibitivo para implementações embarcadas diretas.

\subsection{Restrições Energéticas}

Sistemas embarcados típicos operam com orçamentos energéticos de 5-20W, contrastando drasticamente com os 200-1000W consumidos por servidores de alto desempenho tradicionalmente utilizados para processamento hiperespectral \cite{hwang2011}. Esta limitação torna-se crítica em aplicações UAV onde a autonomia de voo é diretamente afetada pelo consumo energético dos sistemas de processamento.

\section{Lacunas na Literatura}

A análise sistemática de 20 artigos científicos recentes revelou quatro lacunas principais na literatura atual \cite{lou2024, uav_review_2024}:

\begin{enumerate}
\item \textbf{Ausência de Frameworks Integrados}: Embora existam soluções pontuais para otimização específica (GPU, FPGA, ou algoritmos isolados), não há frameworks que integrem sistematicamente múltiplas técnicas de otimização em sistemas heterogêneos.

\item \textbf{Falta de Validação Prática}: A maioria dos trabalhos limita-se a validações em datasets sintéticos ou de laboratório, sem demonstração em aplicações práticas com restrições reais de tempo e energia.

\item \textbf{Trade-offs Não Quantificados}: As relações entre precisão, consumo energético e latência não são adequadamente caracterizadas, dificultando a seleção de técnicas apropriadas para aplicações específicas.

\item \textbf{Metodologias de Codesign Limitadas}: Embora o codesign HW/SW seja reconhecido como fundamental \cite{hwang2011}, faltam metodologias sistemáticas para particionamento otimizado entre CPU, GPU e FPGA em aplicações hiperespectrais.
\end{enumerate}

\section{Escopo da Pesquisa: Estruturação em Duas Etapas}

Esta pesquisa está estruturada em duas etapas complementares, onde a presente dissertação de mestrado representa a primeira etapa focada na validação conceitual e metodológica, preparando o terreno para uma segunda etapa de implementação completa durante o doutorado.

\subsection{Etapa 1: Validação de Metodologias (Mestrado - 2025)}

A primeira etapa, objeto desta dissertação de mestrado, concentra-se na \textbf{validação conceitual e metodológica} das estratégias de integração de sistemas heterogêneos para processamento hiperespectral embarcado. Esta etapa visa estabelecer a base teórica e experimental necessária para o desenvolvimento futuro de uma arquitetura otimizada.

\textbf{Objetivo Principal da Etapa 1}: Validar e quantificar o potencial de integração de técnicas comprovadas (compressive sensing, seleção EMCR, codesign HW/SW) em sistemas heterogêneos para processamento hiperespectral embarcado, gerando análises detalhadas do estado da arte e estabelecendo metodologias de avaliação para orientar o desenvolvimento futuro.

\subsection{Etapa 2: Arquitetura Otimizada (Doutorado - 2026-2029)}

A segunda etapa, a ser desenvolvida durante o doutorado, focará na \textbf{proposição e implementação de uma arquitetura otimizada} baseada nos resultados e validações obtidos na primeira etapa. Esta fase concentrar-se-á no desenvolvimento prático e na inovação arquitetural.

\textbf{Objetivo da Etapa 2}: Propor e implementar uma arquitetura de sistema heterogêneo integrado (CPU+GPU+FPGA) completamente otimizada para redução simultânea de consumo energético e latência no processamento hiperespectral embarcado, baseada nas metodologias validadas na Etapa 1.

\section{Objetivos Específicos da Etapa 1 (Mestrado)}

\begin{enumerate}
\item \textbf{Realizar análise sistemática do estado da arte} em processamento hiperespectral embarcado, identificando e catalogando técnicas comprovadas de otimização energética e redução de latência.

\item \textbf{Caracterizar quantitativamente} os trade-offs entre precisão, consumo energético e latência através de simulações e protótipos conceituais de algoritmos hiperespectrais embarcados.

\item \textbf{Desenvolver metodologia de avaliação} para sistemas heterogêneos, estabelecendo métricas, benchmarks e protocolos de teste para orientar futuras implementações.

\item \textbf{Validar conceitos fundamentais} através de implementações de prova de conceito das técnicas mais promissoras: compressive sensing, seleção EMCR, e codesign básico HW/SW.

\item \textbf{Propor framework arquitetural} para integração sistemática de técnicas em sistemas heterogêneos, definindo especificações e diretrizes para a implementação completa na Etapa 2.

\item \textbf{Estabelecer baseline experimental} através de testes com datasets padrão (AVIRIS, Indian Pines, Pavia) para quantificar o potencial de melhoria e orientar o desenvolvimento futuro.
\end{enumerate}

\section{Hipóteses de Pesquisa da Etapa 1}

Com base na análise bibliográfica sistemática, três hipóteses principais orientam a validação metodológica desta primeira etapa:

\textbf{H1}: A análise sistemática de técnicas comprovadas (compressive sensing com redução 50-70\% dos dados \cite{lim2022}, seleção EMCR com redução 80\% do processamento \cite{martins2019}, e codesign HW/SW com melhoria energética 43.5x \cite{hwang2011}) pode demonstrar, através de simulações e protótipos conceituais, o potencial teórico de redução energética superior a 20x em sistemas hiperespectrais embarcados.

\textbf{H2}: É possível estabelecer, através de modelagem e validação conceitual, que um framework arquitetural heterogêneo com módulos especializados (FPGA para pré-processamento, GPU para reconstrução, CPU para classificação) pode teoricamente atingir metas de latência <50ms/frame mantendo precisão >95\%, fornecendo diretrizes quantitativas para implementação futura.

\textbf{H3}: Uma metodologia sistemática de avaliação e caracterização pode identificar e quantificar os trade-offs fundamentais entre precisão, consumo e latência, estabelecendo um framework de decisão para orientar a seleção e integração de técnicas em diferentes cenários de aplicação na Etapa 2.

\section{Contribuições Esperadas da Etapa 1}

Esta dissertação visa contribuir para o estado da arte em processamento hiperespectral embarcado através de:

\subsection{Contribuições Metodológicas}

\begin{itemize}
\item \textbf{Análise Sistemática Abrangente}: Primeira caracterização quantitativa completa das técnicas de otimização disponíveis, organizando o conhecimento disperso na literatura em um framework coerente.

\item \textbf{Metodologia de Avaliação}: Desenvolvimento de protocolos, métricas e benchmarks padronizados para comparação sistemática de abordagens heterogêneas em processamento hiperespectral embarcado.

\item \textbf{Framework Arquitetural Conceitual}: Proposição de diretrizes e especificações técnicas para integração de técnicas em sistemas heterogêneos, estabelecendo a base teórica para implementações futuras.
\end{itemize}

\subsection{Contribuições Experimentais}

\begin{itemize}
\item \textbf{Validação Conceitual}: Demonstração do potencial teórico através de simulações e protótipos de prova de conceito das técnicas mais promissoras identificadas na literatura.

\item \textbf{Caracterização Quantitativa}: Análise detalhada dos trade-offs precisão vs consumo vs latência usando datasets padrão (AVIRIS, Indian Pines, Pavia University), estabelecendo baselines para comparações futuras.

\item \textbf{Diretrizes de Implementação}: Definição de especificações técnicas, requisitos de hardware e estratégias de integração para orientar o desenvolvimento da arquitetura otimizada na Etapa 2.
\end{itemize}

\section{Metodologia Geral da Etapa 1}

A pesquisa da Etapa 1 seguirá uma abordagem de validação conceitual estruturada em quatro fases principais:

\textbf{Fase 1 - Análise Sistemática do Estado da Arte} (2 meses): Revisão abrangente da literatura, catalogação de técnicas comprovadas, identificação de lacunas e estabelecimento de um framework de conhecimento organizado.

\textbf{Fase 2 - Modelagem e Simulação Conceitual} (3 meses): Desenvolvimento de modelos teóricos, simulações das técnicas mais promissoras e quantificação do potencial de melhoria através de análises computacionais.

\textbf{Fase 3 - Protótipos de Prova de Conceito} (3 meses): Implementação de protótipos simplificados das técnicas principais (compressive sensing, seleção EMCR, codesign básico) para validação experimental dos conceitos fundamentais.

\textbf{Fase 4 - Framework Arquitetural e Diretrizes} (2 meses): Consolidação dos resultados em um framework arquitetural conceitual, definição de especificações técnicas e estabelecimento de diretrizes para a implementação completa na Etapa 2.

\section{Estrutura da Dissertação}

Esta dissertação está organizada de forma direta e objetiva em cinco capítulos focados na validação metodológica:

\textbf{Capítulo 2 - Estado da Arte e Análise de Técnicas} apresenta a análise sistemática da literatura, catalogação de técnicas comprovadas e identificação de lacunas para sistemas heterogêneos em processamento hiperespectral embarcado.

\textbf{Capítulo 3 - Metodologia de Validação} detalha a abordagem de validação conceitual, incluindo modelagem teórica, protocolo de simulação e framework de avaliação para caracterização de trade-offs.

\textbf{Capítulo 4 - Validação Experimental e Resultados} apresenta os protótipos de prova de conceito, simulações das técnicas mais promissoras, análise quantitativa dos trade-offs e estabelecimento de baselines experimentais.

\textbf{Capítulo 5 - Conclusões e Continuidade} sintetiza as contribuições metodológicas da Etapa 1, valida as hipóteses de pesquisa e estabelece as diretrizes técnicas para a implementação da arquitetura otimizada na Etapa 2 (doutorado).

Esta estrutura enxuta garante uma apresentação direta dos objetivos da validação metodológica, maximizando a objetividade e facilitando a transição para a fase de implementação prática no doutorado.
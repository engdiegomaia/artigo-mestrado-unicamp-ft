% Aqui começa o capítulo de Introdução.
% Use o comando \label para definir um rótulo, 
% caso seja necessário referenciar esse capítulo
% posteriormente.
\chapter{Introdução}\label{chp:Introducao}

% Contextualização do problema baseada na nova proposta
O processamento de dados hiperespectrais representa um desafio computacional significativo, especialmente em aplicações de tempo real que demandam alta precisão e eficiência energética. Com o crescimento exponencial da utilização de sensores hiperespectrais em aplicações agrícolas e ambientais, a necessidade por arquiteturas computacionais otimizadas que possam processar grandes volumes de dados espectrais em tempo real tornou-se crítica para viabilizar aplicações práticas efetivas.

O processamento hiperespectral pode ser realizado em diferentes arquiteturas computacionais, cada uma com suas características e trade-offs específicos. Processadores de propósito geral (CPUs) oferecem flexibilidade e facilidade de programação, unidades de processamento gráfico (GPUs) proporcionam paralelismo massivo, e FPGAs (Field-Programmable Gate Arrays) permitem customização de hardware com alta eficiência energética. A escolha da arquitetura mais adequada para uma aplicação específica requer uma análise aprofundada das características e requisitos do processamento hiperespectral em tempo real.

Adicionalmente, a possibilidade de combinar diferentes tecnologias em arquiteturas híbridas emerge como uma alternativa para potencialmente superar limitações individuais de cada processador. A integração estratégica de diferentes arquiteturas pode oferecer oportunidades para otimização de desempenho em cenários específicos, embora também introduza desafios adicionais de complexidade e gerenciamento de recursos.

\section{Motivação}\label{sec:motivacao}

A motivação para esta pesquisa deriva de múltiplos fatores convergentes que destacam a importância de uma análise sistemática e comparativa das diferentes arquiteturas de processamento para aplicações hiperespectrais em tempo real:

\subsection{Desafios Computacionais do Processamento Hiperespectral}
O processamento de dados hiperespectrais apresenta características únicas que demandam soluções computacionais especializadas. Cada pixel hiperespectral contém informações de centenas de bandas espectrais, resultando em volumes de dados que podem exceder terabytes em aplicações típicas. O processamento dessas informações envolve operações computacionalmente intensivas, incluindo correções atmosféricas, redução de dimensionalidade, classificação e análise de padrões espectrais.

Os algoritmos de processamento hiperespectral apresentam diferentes características computacionais que podem se beneficiar de diferentes arquiteturas de processamento. A natureza das operações varia desde cálculos simples e altamente paralelizáveis até algoritmos complexos com dependências de dados significativas, criando um cenário desafiador para a seleção da arquitetura mais apropriada.

\subsection{Características das Arquiteturas de Processamento}
Cada arquitetura de processamento apresenta características distintas que podem ser mais ou menos adequadas para diferentes aspectos do processamento hiperespectral:

\begin{itemize}
    \item \textbf{CPU (Processadores de Propósito Geral)}:
    \begin{itemize}
        \item Flexibilidade e facilidade de programação
        \item Bom desempenho em operações sequenciais
        \item Suporte a instruções vetoriais avançadas
        \item Limitações em paralelismo massivo
    \end{itemize}
    
    \item \textbf{GPU (Unidades de Processamento Gráfico)}:
    \begin{itemize}
        \item Excelente para paralelismo massivo
        \item Alto throughput em operações de ponto flutuante
        \item Otimizado para processamento de dados regulares
        \item Consumo energético significativo
    \end{itemize}
    
    \item \textbf{FPGA (Field-Programmable Gate Arrays)}:
    \begin{itemize}
        \item Alta eficiência energética
        \item Customização de hardware para operações específicas
        \item Excelente para operações de baixa latência
        \item Complexidade de desenvolvimento maior
    \end{itemize}
\end{itemize}

\subsection{Necessidade de Análise Comparativa}
A diversidade de arquiteturas disponíveis e a complexidade das aplicações hiperespectrais criam uma necessidade crítica por:

\begin{itemize}
    \item Avaliação sistemática de desempenho em diferentes cenários
    \item Análise objetiva de trade-offs entre arquiteturas
    \item Compreensão dos impactos em eficiência energética
    \item Identificação de casos de uso ideais para cada arquitetura
\end{itemize}

\section{Objetivos}\label{sec:objetivos}

\subsection{Objetivo Geral}
Realizar uma análise comparativa abrangente de diferentes arquiteturas computacionais (CPU, GPU e FPGA) para processamento de imagens hiperespectrais em tempo real, com foco em aplicações agrícolas e monitoramento ambiental, estabelecendo métricas objetivas de desempenho, eficiência energética e precisão para cada arquitetura em diferentes cenários de aplicação, incluindo uma investigação adicional sobre o potencial de arquiteturas híbridas.

\subsection{Objetivos Específicos}
\begin{enumerate}
    \item \textbf{Caracterizar operações}: Analisar e classificar operações de processamento hiperespectral:
    \begin{itemize}
        \item Complexidade computacional
        \item Requisitos de paralelismo
        \item Padrões de acesso à memória
        \item Dependências de dados
    \end{itemize}
    
    \item \textbf{Implementar algoritmos otimizados}: Desenvolver implementações específicas para cada arquitetura:
    \begin{itemize}
        \item CPU: Otimizações vetoriais e multi-thread
        \item GPU: Implementações CUDA otimizadas
        \item FPGA: Designs VHDL customizados
    \end{itemize}
    
    \item \textbf{Desenvolver framework de simulação}: Criar ambiente controlado para:
    \begin{itemize}
        \item Geração de dados de teste
        \item Medição de métricas
        \item Validação de resultados
        \item Análise comparativa
    \end{itemize}
    
    \item \textbf{Realizar análise comparativa}: Avaliar cada arquitetura em termos de:
    \begin{itemize}
        \item Desempenho computacional
        \item Eficiência energética
        \item Precisão dos resultados
        \item Complexidade de implementação
    \end{itemize}
    
    \item \textbf{Investigar arquiteturas híbridas}: Analisar potenciais benefícios de:
    \begin{itemize}
        \item Combinações de processadores
        \item Estratégias de particionamento
        \item Trade-offs de integração
        \item Cenários de aplicação específicos
    \end{itemize}
    
    \item \textbf{Validar em aplicações reais}: Testar em cenários práticos:
    \begin{itemize}
        \item Datasets hiperespectrais padrão
        \item Aplicações agrícolas específicas
        \item Monitoramento ambiental em tempo real
    \end{itemize}
    
    \item \textbf{Desenvolver diretrizes de seleção}: Estabelecer critérios para:
    \begin{itemize}
        \item Escolha de arquitetura apropriada
        \item Otimizações específicas por cenário
        \item Considerações de implementação
        \item Análise de custo-benefício
    \end{itemize}
\end{enumerate}

\section{Contribuições Esperadas}\label{sec:contribuicoes}

Esta pesquisa visa contribuir para o avanço do estado da arte em processamento hiperespectral através de:

\subsection{Contribuições Metodológicas}
\begin{enumerate}
    \item \textbf{Framework de avaliação}: Metodologia sistemática para:
    \begin{itemize}
        \item Caracterização de operações
        \item Análise de arquiteturas
        \item Medição de desempenho
        \item Comparação objetiva
    \end{itemize}
    
    \item \textbf{Métricas de comparação}: Definição de indicadores para:
    \begin{itemize}
        \item Desempenho computacional
        \item Eficiência energética
        \item Qualidade de resultados
        \item Complexidade de implementação
    \end{itemize}
\end{enumerate}

\subsection{Contribuições Técnicas}
\begin{enumerate}
    \item \textbf{Implementações otimizadas}: Desenvolvimento de:
    \begin{itemize}
        \item Algoritmos CPU vetorizados
        \item Kernels GPU eficientes
        \item Designs FPGA customizados
        \item Integrações híbridas
    \end{itemize}
    
    \item \textbf{Framework de simulação}: Ferramentas para:
    \begin{itemize}
        \item Geração de dados
        \item Medição de métricas
        \item Validação de resultados
        \item Análise comparativa
    \end{itemize}
\end{enumerate}

\subsection{Contribuições Práticas}
\begin{enumerate}
    \item \textbf{Guia de seleção}: Diretrizes para escolha de:
    \begin{itemize}
        \item Arquitetura apropriada
        \item Otimizações específicas
        \item Estratégias de implementação
        \item Considerações de custo
    \end{itemize}
    
    \item \textbf{Análise de viabilidade}: Avaliação de:
    \begin{itemize}
        \item Custo-benefício
        \item Complexidade de desenvolvimento
        \item Requisitos de recursos
        \item Potencial de escalabilidade
    \end{itemize}
\end{enumerate}

\section{Organização do Texto}\label{sec:organizacao}

Esta dissertação está estruturada em sete capítulos:

\begin{itemize}
    \item \textbf{Capítulo 2 - Levantamento Bibliográfico}: Apresenta revisão da literatura sobre processamento hiperespectral, arquiteturas de processamento, técnicas de otimização e aplicações em tempo real.
    
    \item \textbf{Capítulo 3 - Metodologia}: Detalha a metodologia de avaliação, incluindo caracterização de operações, framework de simulação, métricas de avaliação e protocolos experimentais.
    
    \item \textbf{Capítulo 4 - Implementações}: Descreve as implementações otimizadas para cada arquitetura (CPU, GPU, FPGA) e considerações sobre integrações híbridas.
    
    \item \textbf{Capítulo 5 - Resultados}: Apresenta resultados experimentais comparativos, incluindo análises de desempenho, eficiência energética e qualidade.
    
    \item \textbf{Capítulo 6 - Discussão}: Analisa os resultados obtidos, discute trade-offs identificados e propõe diretrizes de seleção de arquitetura.
    
    \item \textbf{Capítulo 7 - Conclusões}: Sumariza as contribuições, limitações e recomendações para trabalhos futuros.
\end{itemize}

\section{Infraestrutura Experimental}\label{sec:infraestrutura}

A validação experimental será realizada utilizando:

\subsection{Plataformas de Desenvolvimento}
\begin{itemize}
    \item \textbf{CPU}: Processador multi-core com suporte AVX-512
    \item \textbf{GPU}: NVIDIA GPU com suporte CUDA
    \item \textbf{FPGA}: Simulação via GHDL e síntese em hardware quando disponível
    \item \textbf{Ambiente}: Linux com ferramentas de desenvolvimento e análise
\end{itemize}

\subsection{Datasets de Validação}
\begin{itemize}
    \item \textbf{Sintéticos}: Datasets gerados para testes controlados
    \item \textbf{Benchmark}: Indian Pines, Pavia University
    \item \textbf{Reais}: Datasets agrícolas e ambientais específicos
\end{itemize}

\subsection{Métricas de Avaliação}
\begin{itemize}
    \item \textbf{Desempenho}: Throughput, latência, utilização de recursos
    \item \textbf{Energia}: Consumo, eficiência, perfil térmico
    \item \textbf{Qualidade}: Precisão, SNR, estabilidade
    \item \textbf{Desenvolvimento}: Complexidade, tempo, manutenibilidade
\end{itemize}

A próxima seção estabelece a fundamentação teórica necessária para compreensão das tecnologias envolvidas e do estado da arte em processamento hiperespectral em tempo real.
% Capitulo de Estrategias de Otimizacao
\chapter{Estrategias de Otimizacao}\label{chp:estrategias}

Este capitulo apresenta as estrategias desenvolvidas para reducao de consumo energetico e latencia no processamento hiperespectral embarcado. As estrategias sao organizadas em tecnicas algoritmicas, otimizacoes de hardware e estrategias de deployment adaptativo, com foco nas aplicacoes praticas identificadas na metodologia.

\section{Estrategias de Reducao de Consumo Energetico}\label{sec:reducao_consumo_estrategias}

A reducao do consumo energetico e abordada atraves de multiplas frentes que trabalham em sinergia para maximizar a eficiencia sem comprometer significativamente a qualidade dos resultados.

\subsection{Otimizacoes Algoritmicas}

\subsubsection{Compressao Adaptativa Espectral}
Desenvolvimento de algoritmos de compressao especificos para dados hiperespectrais que adaptam parametros baseados nas caracteristicas locais:

\paragraph{Compressao Baseada em Correlacao Espectral}
Exploracao da alta correlacao entre bandas adjacentes para reducao de dados processados:
\begin{itemize}
    \item Analise de correlacao em tempo real
    \item Selecao dinamica de bandas representativas
    \item Interpolacao eficiente para reconstrucao
    \item Threshold adaptativo baseado na aplicacao
\end{itemize}

Algoritmo de selecao adaptativa:
\begin{lstlisting}[language=Python]
def adaptive_band_selection(hypercube, correlation_threshold=0.95):
    selected_bands = [0]  # Sempre incluir primeira banda
    for i in range(1, hypercube.shape[2]):
        max_correlation = 0
        for selected in selected_bands:
            corr = correlate_bands(hypercube[:,:,i], 
                                 hypercube[:,:,selected])
            max_correlation = max(max_correlation, corr)
        
        if max_correlation < correlation_threshold:
            selected_bands.append(i)
    
    return selected_bands
\end{lstlisting}

\paragraph{Quantizacao Inteligente}
Implementacao de quantizacao adaptativa que preserva informacoes criticas:
\begin{itemize}
    \item Analise de distribuicao de intensidades por banda
    \item Quantizacao nao-uniforme baseada em histogramas
    \item Preservacao de bordas espectrais criticas
    \item Otimizacao para hardware de precisao limitada
\end{itemize}

\subsubsection{Reducao de Dimensionalidade Energeticamente Eficiente}
Implementacoes otimizadas de tecnicas de reducao dimensional para hardware embarcado:

\paragraph{PCA Incremental para Streaming}
Versao otimizada de PCA que opera em dados streaming com baixo consumo:
\begin{itemize}
    \item Atualizacao incremental de matrizes de covariancia
    \item Aproximacoes numericamente estaveis
    \item Decomposicao singular truncada
    \item Caching inteligente de componentes principais
\end{itemize}

Implementacao incremental:
\begin{lstlisting}[language=Python]
class IncrementalPCA:
    def __init__(self, n_components, batch_size=100):
        self.n_components = n_components
        self.batch_size = batch_size
        self.mean_ = None
        self.components_ = None
        self.n_samples_seen_ = 0
    
    def partial_fit(self, X):
        # Atualizacao incremental eficiente
        if self.mean_ is None:
            self.mean_ = np.mean(X, axis=0)
            self.components_ = self._initial_pca(X)
        else:
            self._update_pca(X)
        
        self.n_samples_seen_ += X.shape[0]
    
    def _update_pca(self, X):
        # Algoritmo de atualizacao O(d^2) ao inves de O(nd^2)
        pass
\end{lstlisting}

\paragraph{Feature Selection Baseada em Entropia}
Selecao de caracteristicas espectrais baseada em criterios de informacao:
\begin{itemize}
    \item Calculo eficiente de entropia espectral
    \item Selecao gulosa de bandas informativas
    \item Balanceamento entre informacao e custo computacional
    \item Adaptacao dinamica aos tipos de alvos
\end{itemize}

\subsubsection{Processamento Hierarquico Adaptativo}
Implementacao de estrategias de processamento em multiplas resolucoes:

\paragraph{Piramide de Processamento}
Arquitetura hierarquica que processa dados em diferentes niveis de detalhe:
\begin{enumerate}
    \item \textbf{Nivel Grosseiro}: Analise rapida de toda a cena
    \item \textbf{Nivel Medio}: Processamento de regioes de interesse
    \item \textbf{Nivel Fino}: Analise detalhada de alvos especificos
\end{enumerate}

\paragraph{Trigger Inteligente}
Sistema de decisao que determina quais regioes necessitam processamento detalhado:
\begin{itemize}
    \item Deteccao de anomalias espectrais
    \item Analise de variancia local
    \item Classificacao preliminar rapida
    \item Threshold adaptativos por aplicacao
\end{itemize}

\subsection{Otimizacoes de Hardware}

\subsubsection{Gerenciamento Dinamico de Energia}
Implementacao de tecnicas que ajustam dinamicamente o consumo baseado na carga:

\paragraph{DVFS (Dynamic Voltage and Frequency Scaling)}
Controle adaptativo de frequencia e voltagem:
\begin{itemize}
    \item Monitoramento continuo de carga computacional
    \item Predicao de demanda baseada em historico
    \item Ajuste automatico de parametros de clock
    \item Minimizacao de overhead de transicao
\end{itemize}

\paragraph{Power Gating Seletivo}
Desligamento seletivo de componentes nao utilizados:
\begin{itemize}
    \item Analise de dependencias entre modulos
    \item Cronograma de ativacao otimizado
    \item Minimizacao de latencia de wake-up
    \item Balanceamento entre economia e disponibilidade
\end{itemize}

\subsubsection{Precisao Aritmetica Adaptativa}
Otimizacao da representacao numerica baseada nos requisitos de precisao:

\paragraph{Mixed-Precision Computing}
Uso de diferentes precisoes para diferentes operacoes:
\begin{itemize}
    \item Analise de sensibilidade por operacao
    \item FP16 para operacoes intermediarias
    \item INT8 para operacoes de baixa sensibilidade
    \item FP32 apenas para calculos criticos
\end{itemize}

\paragraph{Aproximacao Controlada}
Implementacao de aproximacoes matematicas com erro controlado:
\begin{itemize}
    \item Tabelas de lookup para funcoes transcendentais
    \item Aproximacoes polinomiais eficientes
    \item Controle adaptativo de erro
    \item Validacao continua de qualidade
\end{itemize}

\section{Estrategias de Reducao de Latencia}\label{sec:reducao_latencia_estrategias}

A reducao de latencia e crucial para aplicacoes em tempo real e e abordada atraves de paralelizacao eficiente e otimizacoes de pipeline.

\subsection{Paralelizacao Hierarquica}

\subsubsection{Paralelizacao Espectral}
Exploracao do paralelismo natural dos dados hiperespectrais:

\paragraph{Processamento de Bandas Independentes}
Algoritmos que permitem processamento simultaneo de multiplas bandas:
\begin{itemize}
    \item Identificacao de dependencias minimas
    \item Particionamento otimizado do espectro
    \item Sincronizacao eficiente entre threads
    \item Balanceamento de carga dinamico
\end{itemize}

\paragraph{Pipeline Espectral}
Implementacao de pipeline que processa bandas em estagios sobrepostos:
\begin{itemize}
    \item Analise de dependencias temporais
    \item Buffers de sincronizacao otimizados
    \item Predicao de stalls de pipeline
    \item Maximizacao de throughput
\end{itemize}

\subsubsection{Paralelizacao Espacial}
Divisao eficiente do processamento por regioes espaciais:

\paragraph{Tiling Adaptativo}
Divisao inteligente da imagem em tiles para processamento paralelo:
\begin{itemize}
    \item Analise de complexidade local
    \item Tamanho de tile adaptativo
    \item Overlapping otimizado para evitar artefatos
    \item Load balancing baseado em complexidade
\end{itemize}

Implementacao de tiling adaptativo:
\begin{lstlisting}[language=Python]
def adaptive_tiling(image, complexity_map, max_tile_size=256):
    tiles = []
    current_complexity = 0
    current_tile_size = max_tile_size
    
    for region in image_regions:
        local_complexity = complexity_map[region]
        
        # Ajustar tamanho do tile baseado na complexidade
        if local_complexity > threshold_high:
            current_tile_size = max_tile_size // 2
        elif local_complexity < threshold_low:
            current_tile_size = max_tile_size
        
        tiles.append(create_tile(region, current_tile_size))
    
    return tiles
\end{lstlisting}

\paragraph{Streaming Spatial}
Processamento continuo de dados espaciais conforme sao adquiridos:
\begin{itemize}
    \item Buffer circular para dados de entrada
    \item Processamento overlap entre aquisicao e analise
    \item Predicao de proximas regioes de interesse
    \item Minimizacao de latencia sensor-to-output
\end{itemize}

\subsection{Otimizacao de Pipeline}

\subsubsection{Pipeline Assincrono}
Implementacao de pipeline que permite processamento nao-bloqueante:

\paragraph{Multi-Stage Pipeline}
Divisao do processamento em estagios independentes:
\begin{enumerate}
    \item \textbf{Estagio 1}: Aquisicao e pre-processamento basico
    \item \textbf{Estagio 2}: Correcao radiometrica e calibracao
    \item \textbf{Estagio 3}: Reducao de dimensionalidade
    \item \textbf{Estagio 4}: Classificacao e pos-processamento
    \item \textbf{Estagio 5}: Geracao de saida e comunicacao
\end{enumerate}

\paragraph{Balanceamento de Pipeline}
Otimizacao da carga computacional entre estagios:
\begin{itemize}
    \item Analise de gargalos por estagio
    \item Redistribuicao de operacoes
    \item Buffering adaptativo entre estagios
    \item Monitoramento continuo de desempenho
\end{itemize}

\subsubsection{Predicao e Prefetching}
Tecnicas que antecipam necessidades futuras de processamento:

\paragraph{Predicao de Padroes}
Analise de padroes historicos para antecipacao:
\begin{itemize}
    \item Analise temporal de regioes processadas
    \item Predicao de movimentacao de alvos
    \item Antecipacao de mudancas espectrais
    \item Cache inteligente de resultados
\end{itemize}

\paragraph{Prefetching Adaptativo}
Carregamento antecipado de dados baseado em predicoes:
\begin{itemize}
    \item Analise de padroes de acesso
    \item Predicao de proximas requisicoes
    \item Gerenciamento inteligente de cache
    \item Minimizacao de cache misses
\end{itemize}

\section{Deployment Adaptativo}\label{sec:deployment_adaptativo}

O deployment adaptativo permite que o sistema ajuste automaticamente suas estrategias baseado nos recursos disponiveis e requisitos da aplicacao.

\subsection{Configuracao Dinamica}

\subsubsection{Resource-Aware Configuration}
Sistema que ajusta configuracoes baseado nos recursos disponiveis:

\paragraph{Profiling Automatico}
Analise automatica da plataforma de execucao:
\begin{itemize}
    \item Deteccao de tipo de hardware (FPGA, VPU, GPU)
    \item Analise de recursos disponiveis (memoria, computacao)
    \item Medicao de baseline de desempenho
    \item Caracterizacao termica e energetica
\end{itemize}

\paragraph{Configuration Templates}
Templates pre-configurados para diferentes cenarios:
\begin{itemize}
    \item Configuracao para alta precisao
    \item Configuracao para baixo consumo
    \item Configuracao para baixa latencia
    \item Configuracao balanceada
\end{itemize}

\subsubsection{Adaptive Quality Scaling}
Ajuste automatico da qualidade baseado em restricoes:

\paragraph{Quality-Energy Trade-off}
Sistema que balanceava qualidade e consumo:
\begin{itemize}
    \item Monitoramento continuo de energia disponivel
    \item Ajuste dinamico de parametros de qualidade
    \item Feedback loop para otimizacao continua
    \item Garantia de requisitos minimos de qualidade
\end{itemize}

\paragraph{Graceful Degradation}
Estrategias para manter funcionalidade com recursos limitados:
\begin{itemize}
    \item Reducao gradual de complexidade
    \item Fallback para algoritmos mais simples
    \item Processamento seletivo de regioes
    \item Manutencao de funcionalidade critica
\end{itemize}

\subsection{Otimizacao Multi-Objetivo}

\subsubsection{Pareto Optimization}
Encontrar solucoes otimas no espaco multi-dimensional:

\paragraph{Objective Functions}
Definicao de funcoes objetivo balanceadas:
\begin{equation}
F(x) = w_1 \cdot Accuracy(x) - w_2 \cdot Energy(x) - w_3 \cdot Latency(x)
\end{equation}
onde $w_i$ sao pesos adaptativos baseados na aplicacao.

\paragraph{Dynamic Weight Adjustment}
Ajuste dinamico de pesos baseado em condicoes operacionais:
\begin{itemize}
    \item Priorizacao baseada em contexto
    \item Adaptacao a mudancas de requisitos
    \item Learning de preferencias do usuario
    \item Otimizacao continua
\end{itemize}

\subsubsection{Reinforcement Learning para Otimizacao}
Uso de aprendizado por reforco para otimizacao continua:

\paragraph{Q-Learning para Resource Management}
Sistema que aprende politicas otimas de gerenciamento:
\begin{itemize}
    \item Estado: recursos disponiveis, carga atual, requisitos
    \item Acoes: configuracoes de sistema disponiveis
    \item Recompensa: funcao multi-objetivo balanceada
    \item Politica: estrategia otima aprendida
\end{itemize}

\paragraph{Online Learning}
Adaptacao continua baseada em experiencia operacional:
\begin{itemize}
    \item Coleta continua de metricas de desempenho
    \item Atualizacao de modelos em tempo real
    \item Adaptacao a mudancas de ambiente
    \item Melhoria continua de eficiencia
\end{itemize}

\section{Implementacao Especifica por Aplicacao}\label{sec:implementacao_aplicacao}

As estrategias sao adaptadas as caracteristicas especificas de cada aplicacao alvo.

\subsection{Agricultura de Precisao}

\subsubsection{Deteccao de Estresse em Culturas}
Otimizacoes especificas para identificacao de problemas agricolas:

\paragraph{Bandas Espectrais Criticas}
Foco em faixas espectrais mais informativas:
\begin{itemize}
    \item Red-edge (700-750nm) para estado fotossintetico
    \item NIR (750-900nm) para estrutura foliar
    \item SWIR (1200-1600nm) para conteudo de agua
    \item Indices espectrais especializados (NDVI, GNDVI, MCARI)
\end{itemize}

\paragraph{Classificacao Hierarquica}
Sistema de classificacao em multiplos niveis:
\begin{enumerate}
    \item Deteccao de vegetacao vs. solo
    \item Classificacao de tipos de cultura
    \item Identificacao de condicoes de estresse
    \item Quantificacao de severidade
\end{enumerate}

\subsubsection{Monitoramento Temporal}
Otimizacoes para acompanhamento ao longo do tempo:

\paragraph{Change Detection Eficiente}
Algoritmos otimizados para deteccao de mudancas:
\begin{itemize}
    \item Comparacao com baseline historico
    \item Filtros temporais para reducao de ruido
    \item Deteccao de tendencias espectrais
    \item Alerta automatico para anomalias
\end{itemize}

\subsection{Monitoramento Ambiental}

\subsubsection{Deteccao de Queimadas}
Estrategias especificas para identificacao de focos de incendio:

\paragraph{Multi-spectral Fire Detection}
Combinacao otimizada de bandas para deteccao:
\begin{itemize}
    \item Thermal infrared (8-14 um) para temperatura
    \item SWIR (1.6-2.4 um) para gases de combustao
    \item Visible (0.4-0.7 um) para fumaca
    \item Algoritmos de fusao multi-espectral
\end{itemize}

\paragraph{Real-time Alert System}
Sistema de alerta com latencia ultra-baixa:
\begin{itemize}
    \item Processamento prioritario de regioes criticas
    \item Algoritmos de deteccao simplificados
    \item Sistema de comunicacao de emergencia
    \item Validacao cruzada automatica
\end{itemize}

\subsection{Sistemas de Vigilancia}

\subsubsection{Reconhecimento de Alvos}
Otimizacoes para identificacao e classificacao:

\paragraph{Target vs. Background Separation}
Algoritmos otimizados para distincao alvo-fundo:
\begin{itemize}
    \item Analise de assinaturas espectrais unicas
    \item Deteccao de anomalias espectrais
    \item Classificacao supervisionada adaptativa
    \item Fusao com informacoes contextuais
\end{itemize}

\paragraph{Stealth Detection}
Tecnicas para identificacao de objetos camuflados:
\begin{itemize}
    \item Analise de inconsistencias espectrais
    \item Deteccao de bordas espectrais anomalas
    \item Comparacao com modelos naturais
    \item Fusao temporal para deteccao de movimento
\end{itemize}

O proximo capitulo apresenta a implementacao pratica dessas estrategias em hardware embarcado e os resultados de simulacao via GHDL. 
%% Capítulo 3: Metodologia
%% Baseado na Arquitetura de Sistema Heterogêneo Proposta

\section{Visão Geral da Metodologia}

Esta pesquisa propõe uma metodologia estruturada em quatro fases para desenvolvimento de um sistema heterogêneo otimizado para processamento hiperespectral embarcado. A abordagem integra técnicas comprovadas de otimização (identificadas na revisão bibliográfica) em uma arquitetura CPU+GPU+FPGA com pipeline especializado para redução simultânea de consumo energético e latência.

A metodologia baseia-se em três pilares fundamentais: \textbf{(1)} profiling sistemático para identificação de gargalos computacionais, \textbf{(2)} implementação de módulos especializados com técnicas de otimização comprovadas, e \textbf{(3)} integração heterogênea com balanceamento dinâmico de carga e gestão inteligente de energia.

\section{Arquitetura do Sistema Proposto}

\subsection{Visão Sistêmica}

O sistema heterogêneo proposto implementa um pipeline de três estágios especializados:

\begin{enumerate}
\item \textbf{Módulo FPGA}: Pré-processamento especializado incluindo correção radiométrica ELM, seleção de bandas EMCR, e encoding compressive sensing
\item \textbf{Módulo GPU}: Reconstrução CGNE, extração de características com CNNs 3D otimizadas, e fusão espacial-espectral
\item \textbf{Módulo CPU}: Classificação SVM com distância de Hamming, controle adaptativo de qualidade, e coordenação do sistema
\end{enumerate}

\subsection{Plataforma Hardware}

A plataforma base selecionada combina:
\begin{itemize}
\item \textbf{CPU}: ARM Cortex-A78 para controle e algoritmos adaptativos
\item \textbf{GPU}: NVIDIA Jetson Orin Nano (15W TDP) para processamento paralelo
\item \textbf{FPGA}: Xilinx Zynq UltraScale+ para pré-processamento especializado
\item \textbf{Memória}: 16GB LPDDR5 compartilhada entre módulos
\item \textbf{Armazenamento}: SSD NVMe 256GB para datasets e modelos
\end{itemize}

A seleção desta plataforma baseia-se na validação experimental de Díaz et al. \cite{diaz2019} para GPUs embarcadas e Hwang et al. \cite{hwang2011} para codesign FPGA, oferecendo a melhor relação performance/consumo para aplicações hiperespectrais.

\section{Fase 1: Análise e Baseline (2 meses)}

\subsection{Profiling Sistemático}

Seguindo a metodologia de Hwang et al. \cite{hwang2011}, o profiling sistemático visa identificar os gargalos computacionais que consomem >90\% do tempo de processamento:

\textbf{Objetivos Específicos}:
\begin{itemize}
\item Análise de tempo de execução por estágio do pipeline hiperespectral
\item Medição de bandwidth e padrões de acesso à memória
\item Profiling energético detalhado por módulo computacional
\item Identificação de oportunidades de paralelização
\end{itemize}

\textbf{Ferramentas de Profiling}:
\begin{itemize}
\item \textbf{NVIDIA NSight Systems}: Análise detalhada de kernels CUDA e utilização GPU
\item \textbf{Intel VTune Profiler}: Profiling CPU incluindo cache misses e utilização SIMD
\item \textbf{Xilinx Vivado HLS}: Análise de síntese e profiling FPGA
\item \textbf{PowerTOP/Energymon}: Medição de consumo energético por subsistema
\end{itemize}

\textbf{Metodologia de Análise}:
\begin{enumerate}
\item Implementação baseline single-core CPU dos algoritmos principais
\item Profiling detalhado identificando top-5 gargalos computacionais
\item Análise de dependências entre estágios e oportunidades de pipeline
\item Caracterização de padrões de acesso à memória e bandwidth requirements
\item Medição de baseline energético e estabelecimento de métricas de referência
\end{enumerate}

\subsection{Seleção de Datasets e Métricas}

\textbf{Datasets de Validação}:
Baseados nas referências da literatura \cite{lou2024, ullah2020}:
\begin{itemize}
\item \textbf{AVIRIS Indian Pines}: Dataset padrão (145×145×220) para agricultura
\item \textbf{Pavia University}: Ambiente urbano (610×340×103) para robustez
\item \textbf{Salinas Valley}: Agricultura diversificada (512×217×224)
\item \textbf{Dataset UAV próprio}: Validação prática em cenário real controlado
\end{itemize}

\textbf{Métricas de Avaliação}:
\begin{itemize}
\item \textbf{Performance}: Throughput (fps), latência (ms/pixel), utilização recursos (\%)
\item \textbf{Energética}: Consumo total (W), eficiência energética (fps/W), autonomia estimada
\item \textbf{Qualidade}: Precisão (\%), recall (\%), F1-score, kappa coefficient
\item \textbf{Sistema}: Temperatura (°C), overhead comunicação (\%), escalabilidade
\end{itemize}

\section{Fase 2: Implementação de Módulos Core (4 meses)}

\subsection{Módulo FPGA - Pré-processamento Especializado}

O módulo FPGA implementa três técnicas de otimização sequenciais baseadas na literatura:

\subsubsection{Correção Radiométrica ELM}

Implementação do Empirical Line Method conforme Shin et al. \cite{shin2024}:

\begin{lstlisting}[language=verilog, caption=Módulo ELM em VHDL]
entity empirical_line_correction is
    generic (
        PIXEL_WIDTH : natural := 16;
        BAND_COUNT  : natural := 224
    );
    port (
        clk              : in  std_logic;
        rst              : in  std_logic;
        raw_pixel        : in  std_logic_vector(PIXEL_WIDTH-1 downto 0);
        dark_reference   : in  std_logic_vector(PIXEL_WIDTH-1 downto 0);
        bright_reference : in  std_logic_vector(PIXEL_WIDTH-1 downto 0);
        corrected_pixel  : out std_logic_vector(PIXEL_WIDTH-1 downto 0);
        valid_out        : out std_logic
    );
end entity;
\end{lstlisting}

\textbf{Justificativa}: Shin et al. demonstraram melhoria de 5-55\% na reflectância com baixo custo computacional, sendo adequado para implementação FPGA.

\subsubsection{Seleção de Bandas EMCR}

Implementação do método EMCR conforme Martins et al. \cite{martins2019}:

\begin{lstlisting}[language=verilog, caption=Seleção EMCR em VHDL]
entity emcr_band_selection is
    generic (
        INPUT_BANDS  : natural := 224;
        OUTPUT_BANDS : natural := 45  -- 80% reduction
    );
    port (
        clk            : in  std_logic;
        spectral_bands : in  pixel_array(INPUT_BANDS-1 downto 0);
        selected_bands : out pixel_array(OUTPUT_BANDS-1 downto 0);
        band_indices   : out index_array(OUTPUT_BANDS-1 downto 0);
        selection_done : out std_logic
    );
end entity;
\end{lstlisting}

\textbf{Justificativa}: Redução comprovada de 80\% do processamento mantendo 99.7\% de precisão, essencial para viabilizar tempo real.

\subsubsection{Compressive Sensing Encoder}

Implementação de encoder CS baseada em Lim et al. \cite{lim2022}:

\begin{lstlisting}[language=verilog, caption=CS Encoder em VHDL]
entity cs_encoder is
    generic (
        INPUT_SIZE      : natural := 45;
        COMPRESSED_SIZE : natural := 23  -- 50% compression
    );
    port (
        clk                  : in  std_logic;
        input_bands          : in  pixel_array(INPUT_SIZE-1 downto 0);
        sensing_matrix       : in  matrix_type;
        compressed_data      : out pixel_array(COMPRESSED_SIZE-1 downto 0);
        sparse_indices       : out index_array;
        compression_complete : out std_logic
    );
end entity;
\end{lstlisting}

\textbf{Recursos FPGA Estimados}:
\begin{itemize}
\item DSPs: 45\% (600 de 1340 disponíveis no UltraScale+)
\item BRAMs: 60\% (850 de 1418 disponíveis)
\item LUTs: 70\% (240K de 343K disponíveis)
\end{itemize}

\subsection{Módulo GPU - Reconstrução e Extração de Características}

O módulo GPU implementa duas funções principais usando CUDA otimizado:

\subsubsection{Reconstrução Compressive Sensing}

Implementação CGNE baseada em Lim et al. \cite{lim2022}:

\begin{lstlisting}[language=c, caption=CGNE Solver em CUDA]
__global__ void cgne_reconstruction_kernel(
    const float* compressed_data,     // Compressed data from FPGA
    const float* sensing_matrix,      // Sensing matrix
    float* reconstructed_image,       // Reconstructed image
    const int compressed_size,
    const int original_size,
    const int max_iterations
) {
    int tid = blockIdx.x * blockDim.x + threadIdx.x;
    
    // CGNE implementation with sparsity exploration
    // Optimized for Pascal/Ampere architecture
    if (tid < original_size) {
        // Parallelized CGNE algorithm
        // Matrix sparsity exploration
        // Adaptive convergence
    }
}
\end{lstlisting}

\subsubsection{Extração de Características CNN 3D}

Implementação baseada em TakuNet \cite{takunet2025} e UAV Review \cite{uav_review_2024}:

\begin{lstlisting}[language=c, caption=CNN 3D Otimizada]
__global__ void depthwise_conv3d_kernel(
    const float* input_data,          // Reconstructed data
    const float* weights,             // Depth-wise weights
    float* output_features,           // Extracted features
    const int bands, 
    const int height, 
    const int width,
    const int kernel_size
) {
    int tid = blockIdx.x * blockDim.x + threadIdx.x;
    
    // Depth-wise convolution 3D implementation
    // Optimized for hyperspectral data
    // FP16 precision for efficiency
}
\end{lstlisting}

\textbf{Otimizações GPU Específicas}:
\begin{itemize}
\item \textbf{FP16 Precision}: Redução de 50\% no uso de memória
\item \textbf{Tensor Cores}: Utilização de unidades especializadas para deep learning
\item \textbf{Memory Coalescing}: Otimização de padrões de acesso à memória
\item \textbf{Warp-level Primitives}: Sincronização eficiente entre threads
\end{itemize}

\subsection{Módulo CPU - Classificação e Controle}

O módulo CPU implementa classificação final e controle do sistema:

\subsubsection{Classificador SVM Otimizado}

Baseado em Martins et al. \cite{martins2019}:

\begin{lstlisting}[language=c++, caption=SVM com Distância de Hamming]
class OptimizedSVM {
private:
    std::vector<float> support_vectors;
    std::vector<int> binary_features;
    
public:
    void hamming_distance_classification(
        const std::vector<float>& features,
        std::vector<uint8_t>& classification
    ) {
        // Convert to binary representation
        auto binary_features = convert_to_binary(features);
        
        // Classification using Hamming distance
        // Optimized for ARM NEON SIMD instructions
        for (size_t i = 0; i < features.size(); ++i) {
            classification[i] = hamming_classify(binary_features[i]);
        }
    }
};
\end{lstlisting}

\subsubsection{Controle Adaptativo de Qualidade}

\begin{lstlisting}[language=c++, caption=Controle Adaptativo]
class AdaptiveQualityController {
private:
    float current_power_budget;
    float target_latency;
    QualityMetrics current_metrics;
    
public:
    void adjust_system_parameters() {
        // Dynamic adjustment based on feedback
        if (current_metrics.latency > target_latency) {
            reduce_quality_settings();
        }
        
        if (current_power_consumption > current_power_budget) {
            enable_power_saving_mode();
        }
        
        // Automatic quality vs resources balancing
        optimize_tradeoffs();
    }
};
\end{lstlisting}

\section{Fase 3: Integração e Otimização do Sistema (3 meses)}

\subsection{Pipeline Heterogêneo}

A integração dos módulos segue arquitetura de pipeline com overlap de processamento:

\begin{lstlisting}[language=c++, caption=Pipeline Heterogêneo]
class HeterogeneousPipeline {
private:
    FPGAProcessor fpga_module;
    GPUProcessor gpu_module;
    CPUClassifier cpu_module;
    
    // Lock-free queues for communication
    LockFreeQueue<PreprocessedData> fpga_to_gpu;
    LockFreeQueue<FeatureData> gpu_to_cpu;
    
public:
    void process_frame(const HyperspectralFrame& frame) {
        // Stage 1: FPGA preprocessing (pipeline)
        auto preprocessed = fpga_module.preprocess_async(frame);
        
        // Stage 2: GPU reconstruction/extraction (overlap)
        auto features = gpu_module.extract_features_async(preprocessed);
        
        // Stage 3: CPU classification (overlap)
        auto results = cpu_module.classify_async(features);
        
        // Feedback loop for continuous optimization
        update_system_parameters(results.quality_metrics);
    }
};
\end{lstlisting}

\subsection{Sincronização e Comunicação}

\textbf{Estratégias de Comunicação}:
\begin{itemize}
\item \textbf{DMA Transfers}: Comunicação FPGA-GPU sem intervenção CPU
\item \textbf{CUDA Streams}: Overlap de computation e communication
\item \textbf{Lock-free Queues}: Comunicação CPU-GPU sem bloqueios
\item \textbf{Memory Mapping}: Compartilhamento eficiente de buffers grandes
\end{itemize}

\subsection{Balanceamento de Carga Dinâmico}

\begin{lstlisting}[language=c++, caption=Balanceamento Dinâmico]
class LoadBalancer {
private:
    SystemMetrics current_metrics;
    
public:
    void distribute_workload() {
        auto utilization = measure_utilization();
        
        if (utilization.fpga < 0.7 && utilization.gpu > 0.9) {
            migrate_preprocessing_to_fpga();
        }
        
        if (utilization.cpu < 0.5 && utilization.gpu > 0.8) {
            offload_simple_operations_to_cpu();
        }
        
        adjust_pipeline_parameters(utilization);
    }
};
\end{lstlisting}

\section{Fase 4: Validação e Otimização Avançada (2 meses)}

\subsection{Cenários de Validação Experimental}

\textbf{Cenário 1 - Agricultura de Precisão}:
\begin{itemize}
\item UAV a 100m de altitude com sensor hiperespectral 224 bandas
\item Área de teste: 1 hectare de cultura diversificada
\item Aplicação: Detecção de estresse hídrico e nutricional
\item Requisitos: 25 fps, autonomia >30 min, precisão >95\%
\end{itemize}

\textbf{Cenário 2 - Monitoramento Urbano}:
\begin{itemize}
\item Classificação LULC em tempo real
\item Processamento contínuo de 30 fps por 2 horas
\item Aplicação: Monitoramento de crescimento urbano
\item Requisitos: Latência <100ms, consumo <20W
\end{itemize}

\textbf{Cenário 3 - Aplicação de Emergência}:
\begin{itemize}
\item Detecção de incêndios florestais
\item Processamento em condições de alta temperatura
\item Aplicação: Alerta precoce de incêndios
\item Requisitos: Latência crítica <50ms, robustez térmica
\end{itemize}

\subsection{Protocolo Experimental}

\textbf{Metodologia de Comparação}:
\begin{enumerate}
\item Baseline single-core CPU (referência)
\item Implementação GPU pura (estado da arte)
\item Sistema heterogêneo proposto (completo)
\item Comparação com trabalhos relacionados (Díaz et al., Hwang et al.)
\end{enumerate}

\textbf{Métricas de Validação}:
\begin{itemize}
\item \textbf{Performance}: Throughput absoluto e latência por estágio
\item \textbf{Eficiência Energética}: mW/fps e autonomia estimada
\item \textbf{Precisão}: Acurácia, F1-score e matriz de confusão
\item \textbf{Robustez}: Variação de performance em condições adversas
\end{itemize}

\section{Metas de Performance Estabelecidas}

Baseadas na análise bibliográfica e capacidades das plataformas selecionadas:

\begin{table}[H]
\centering
\caption{Metas de Performance do Sistema Proposto}
\begin{tabular}{|l|c|c|c|}
\hline
\textbf{Métrica} & \textbf{Target} & \textbf{Baseline} & \textbf{Melhoria} \\
\hline
Throughput & 100 fps & 15 fps & 6.7x \\
Latência & <50ms & 200ms & 4x \\
Consumo Energético & 15W & 45W & 3x \\
Precisão & >95\% & 92\% & +3\% \\
Taxa de Compressão & 10:1 & 3:1 & 3.3x \\
\hline
\end{tabular}
\label{tab:performance_targets}
\end{table}

\textbf{Justificativas das Metas}:
\begin{itemize}
\item \textbf{Throughput}: Baseado em 330 fps (Díaz et al.) com margem para overhead de integração
\item \textbf{Latência}: Combinação de <10ms (Lim et al.) + 0.1ms/pixel (Martins et al.)
\item \textbf{Consumo}: Target conservador baseado em 43.5x (Hwang et al.) em plataforma embarcada
\item \textbf{Precisão}: Manutenção do nível de 99.7\% (Martins et al.) com recursos limitados
\end{itemize}

\section{Contribuições Metodológicas Esperadas}

\subsection{Framework de Codesign Sistemático}

Desenvolvimento de metodologia reproduzível para particionamento HW/SW baseada em:
\begin{itemize}
\item Profiling automatizado e identificação de gargalos
\item Análise quantitativa de trade-offs precisão vs recursos
\item Algoritmos de decisão para seleção de implementação otimizada
\end{itemize}

\subsection{Algoritmos Adaptativos}

Implementação de controle adaptativo que ajusta dinamicamente:
\begin{itemize}
\item Precisão de processamento (FP32/FP16/INT8) conforme orçamento energético
\item Taxa de compressão baseada em complexidade da cena
\item Balanceamento de carga entre módulos conforme utilização
\end{itemize}

\subsection{Métricas e Benchmarks}

Estabelecimento de métricas padronizadas para:
\begin{itemize}
\item Eficiência energética normalizada (fps/W/precisão)
\item Latência por estágio em pipelines heterogêneos
\item Trade-off curves precisão vs recursos para diferentes aplicações
\end{itemize}

Esta metodologia fornece uma abordagem sistemática e reproduzível para desenvolvimento de sistemas hiperespectrais embarcados otimizados, integrando as melhores técnicas identificadas na literatura em um framework unificado com validação experimental robusta.
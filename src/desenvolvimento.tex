% Capítulo de Metodologia e Caracterização de Datasets
\chapter{Metodologia e Caracterização de Datasets}\label{chp:metodologia}

Este capítulo apresenta a metodologia desenvolvida para caracterização e seleção de datasets hiperespectrais que simulem fielmente operações embarcadas reais, bem como o framework de avaliação de estratégias para redução de consumo energético e latência. A metodologia proposta estabelece uma abordagem sistemática para validação de sistemas hiperespectrais embarcados em aplicações práticas específicas.

\section{Visão Geral da Metodologia}\label{sec:visao_geral}

A metodologia é fundamentada na premissa de que a validação efetiva de sistemas hiperespectrais embarcados requer datasets que representem fielmente as condições operacionais reais, incluindo variabilidade ambiental, características dos alvos e restrições de sistema.

% Figura da metodologia será incluída posteriormente
% \begin{figure}[!htb]
% \centering
% \includegraphics[width=0.9\textwidth]{metodologia_datasets.png}
% \caption[Metodologia de Caracterização]{Metodologia proposta para caracterização e seleção de datasets para sistemas embarcados.}
% \label{fig:metodologia_datasets}
% \end{figure}

A metodologia é estruturada em seis componentes principais:

\begin{enumerate}
    \item \textbf{Definição de Cenários e Aplicações-alvo}: Caracterização detalhada das situações reais
    \item \textbf{Processo de Seleção de Datasets}: Critérios sistemáticos para escolha
    \item \textbf{Preparação e Adaptação}: Técnicas para simulação de operações reais
    \item \textbf{Framework de Simulação Embarcada}: Ambiente controlado para validação
    \item \textbf{Métricas de Avaliação}: Indicadores específicos para sistemas embarcados
    \item \textbf{Validação em Aplicações Práticas}: Teste em cenários operacionais
\end{enumerate}

\section{Definição de Cenários e Aplicações-alvo}\label{sec:cenarios_aplicacoes}

O primeiro passo metodológico consiste na definição detalhada dos cenários de aplicação que se deseja simular, estabelecendo parâmetros operacionais específicos para cada caso de uso.

\subsection{Agricultura de Precisão}

\subsubsection{Detecção de Estresse em Lavouras}
\begin{itemize}
    \item \textbf{Características Ambientais}:
    \begin{itemize}
        \item Variação de iluminação: 200-2000 W/m² (condições de nascer do sol a meio-dia)
        \item Umidade relativa: 40-90\%
        \item Temperatura operacional: -10°C a 50°C
        \item Presença de poeira e particulados
    \end{itemize}
    
    \item \textbf{Restrições Operacionais}:
    \begin{itemize}
        \item Processamento onboard em drone/robô agrícola
        \item Autonomia energética: 2-8 horas de operação contínua
        \item Taxa de aquisição: 10-30 fps
        \item Latência máxima: 100ms para decisão de aplicação
        \item Resolução espacial: 1-10 cm/pixel
        \item Resolução espectral: 10-400 bandas (400-1000 nm)
    \end{itemize}
    
    \item \textbf{Alvos de Detecção}:
    \begin{itemize}
        \item Deficiências nutricionais (N, P, K)
        \item Estresse hídrico
        \item Doenças foliares
        \item Infestação de pragas
        \item Maturação de frutos
    \end{itemize}
\end{itemize}

\subsubsection{Monitoramento de Crescimento}
\begin{itemize}
    \item \textbf{Requisitos Temporais}: Aquisições periódicas (diárias/semanais)
    \item \textbf{Consistência Espectral}: Calibração mantida ao longo da safra
    \item \textbf{Cobertura Espacial}: Campos de 10-1000 hectares
\end{itemize}

\subsection{Monitoramento Ambiental}

\subsubsection{Detecção de Queimadas}
\begin{itemize}
    \item \textbf{Características Ambientais}:
    \begin{itemize}
        \item Condições de baixa visibilidade (fumaça)
        \item Variações térmicas extremas
        \item Operação 24/7 em locais remotos
    \end{itemize}
    
    \item \textbf{Restrições Operacionais}:
    \begin{itemize}
        \item Sistema autônomo com energia solar/bateria
        \item Conectividade limitada (satelital)
        \item Latência crítica: <30s para alerta
        \item Operação por meses sem manutenção
    \end{itemize}
    
    \item \textbf{Alvos de Detecção}:
    \begin{itemize}
        \item Focos iniciais de combustão
        \item Progressão de incêndios
        \item Áreas queimadas
        \item Recuperação pós-incêndio
    \end{itemize}
\end{itemize}

\subsubsection{Qualidade da Água}
\begin{itemize}
    \item \textbf{Monitoramento Contínuo}: Sensores fixos em corpos d'água
    \item \textbf{Variabilidade Sazonal}: Adaptação a mudanças sazonais
    \item \textbf{Detecção de Poluição}: Identificação de contaminantes específicos
\end{itemize}

\subsection{Sistemas de Vigilância}

\subsubsection{Reconhecimento de Alvos}
\begin{itemize}
    \item \textbf{Características Ambientais}:
    \begin{itemize}
        \item Operação dia/noite
        \item Condições meteorológicas variadas
        \item Camuflagem natural/artificial
    \end{itemize}
    
    \item \textbf{Restrições Operacionais}:
    \begin{itemize}
        \item Operação discreta (baixo consumo)
        \item Resposta imediata (<50ms)
        \item Falsas alarmes mínimas
        \item Autonomia energética estendida
    \end{itemize}
    
    \item \textbf{Alvos de Interesse}:
    \begin{itemize}
        \item Pessoas vs. animais
        \item Veículos específicos
        \item Materiais contrabandeados
        \item Alterações na paisagem
    \end{itemize}
\end{itemize}

\section{Processo Detalhado de Seleção de Datasets}\label{sec:selecao_datasets}

A seleção de datasets segue um protocolo sistemático que garante representatividade e adequação aos cenários definidos.

\subsection{Pesquisa e Levantamento de Datasets Existentes}

\subsubsection{Repositórios Consultados}
\begin{itemize}
    \item \textbf{Agricultura}: 
    \begin{itemize}
        \item Indian Pines, Salinas Valley, Pavia University
        \item Kennedy Space Center, Botswana
        \item AVIRIS datasets agrícolas
    \end{itemize}
    
    \item \textbf{Ambiental}:
    \begin{itemize}
        \item AVIRIS (Airborne Visible/Infrared Imaging Spectrometer)
        \item ROSIS (Reflective Optics System Imaging Spectrometer)
        \item Datasets da NASA, ESA, INPE
        \item EO-1 Hyperion archive
    \end{itemize}
    
    \item \textbf{Industrial/Vigilância}:
    \begin{itemize}
        \item CAVE multispectral database
        \item BGU Hyperspectral database
        \item Datasets de inspeção industrial
    \end{itemize}
    
    \item \textbf{Personalizados}:
    \begin{itemize}
        \item Coletas específicas de grupos de pesquisa
        \item Campanhas de campo direcionadas
        \item Datasets sintéticos gerados
    \end{itemize}
\end{itemize}

\subsubsection{Critérios de Avaliação Inicial}
Para cada dataset identificado, verificamos:
\begin{itemize}
    \item Faixa espectral coberta e número de bandas
    \item Resolução espacial e área coberta
    \item Presença de ground truth/anotações
    \item Metadados de aquisição (condições, calibração)
    \item Formato de dados e acessibilidade
    \item Licença de uso e restrições
\end{itemize}

\subsection{Critérios de Seleção Detalhados}

A seleção final baseia-se em cinco critérios principais com pesos específicos:

\subsubsection{Correspondência com Cenário Real (Peso: 30\%)}
\begin{itemize}
    \item \textbf{Similaridade de Alvos}: Presença dos materiais/condições de interesse
    \item \textbf{Condições de Aquisição}: Compatibilidade com cenário operacional
    \item \textbf{Escala Espacial}: Correspondência com área de operação típica
    \item \textbf{Resolução Adequada}: Espacial e espectral apropriadas
\end{itemize}

Métrica de avaliação:
\begin{equation}
S_{corresp} = \frac{1}{4} \sum_{i=1}^{4} w_i \cdot s_i
\end{equation}
onde $s_i$ são scores individuais (0-1) para cada subcritério e $w_i$ são pesos específicos por aplicação.

\subsubsection{Diversidade de Condições (Peso: 25\%)}
\begin{itemize}
    \item \textbf{Variabilidade Temporal}: Diferentes épocas, condições sazonais
    \item \textbf{Variabilidade Espacial}: Diferentes locais, biomas
    \item \textbf{Condições Ambientais}: Iluminação, clima, atmosfera
    \item \textbf{Estado dos Alvos}: Diferentes estágios, condições
\end{itemize}

Métrica de diversidade baseada em entropia espectral:
\begin{equation}
D_{cond} = -\sum_{j=1}^{N} p_j \log_2(p_j)
\end{equation}
onde $p_j$ representa a probabilidade de cada condição ambiental no dataset.

\subsubsection{Completeness e Qualidade (Peso: 20\%)}
\begin{itemize}
    \item \textbf{Ground Truth}: Precisão e cobertura das anotações
    \item \textbf{Metadados}: Completude das informações auxiliares
    \item \textbf{Calibração}: Qualidade radiométrica dos dados
    \item \textbf{Integridade}: Ausência de gaps ou corrupções
\end{itemize}

\subsubsection{Volume e Representatividade Estatística (Peso: 15\%)}
\begin{itemize}
    \item \textbf{Tamanho Adequado}: Suficiente para validação robusta
    \item \textbf{Distribuição Balanceada}: Representação adequada de classes
    \item \textbf{Significância Estatística}: Amostras suficientes por categoria
\end{itemize}

\subsubsection{Acessibilidade e Usabilidade (Peso: 10\%)}
\begin{itemize}
    \item \textbf{Licença Aberta}: Disponibilidade para pesquisa
    \item \textbf{Formato Padrão}: Compatibilidade com ferramentas
    \item \textbf{Documentação}: Clareza das especificações
    \item \textbf{Facilidade de Download}: Acessibilidade técnica
\end{itemize}

\subsection{Matriz de Decisão}
A seleção final utiliza uma matriz de decisão que combina todos os critérios:

\begin{table}[!htp]
\caption[Matriz de Avaliação de Datasets]{Matriz de avaliação para seleção de datasets.}
\label{tab:matriz_datasets}
\begin{center}
\begin{tabular}{|p{3cm}|p{2cm}|p{2cm}|p{2cm}|p{2cm}|}
\hline
\textbf{Dataset} & \textbf{Correspondência} & \textbf{Diversidade} & \textbf{Qualidade} & \textbf{Score Final} \\
\hline
Indian Pines & 0.85 & 0.70 & 0.90 & 0.81 \\
\hline
Salinas Valley & 0.90 & 0.75 & 0.85 & 0.83 \\
\hline
Pavia University & 0.75 & 0.80 & 0.95 & 0.82 \\
\hline
\end{tabular}
\end{center}
\end{table}

\section{Preparação e Adaptação dos Datasets}\label{sec:preparacao_dados}

Após a seleção, os datasets são preparados para simular condições operacionais embarcadas reais.

\subsection{Pré-processamento para Operação Embarcada}

\subsubsection{Correção Radiométrica Simplificada}
Implementação de correções radiométricas otimizadas para hardware embarcado:
\begin{itemize}
    \item Algoritmos de correção de ganho e offset simplificados
    \item Tabelas de lookup para correção atmosférica
    \item Normalização adaptativa baseada em estatísticas locais
\end{itemize}

\subsubsection{Formato de Streaming}
Conversão dos datasets para formato que simule aquisição em tempo real:
\begin{itemize}
    \item Segmentação em blocos temporais (line-scan simulation)
    \item Buffering adaptativo baseado em memória disponível
    \item Sincronização com taxas de aquisição realísticas
\end{itemize}

\subsubsection{Simulação de Limitações de Hardware}
Introdução de restrições típicas de sistemas embarcados:
\begin{itemize}
    \item Redução de precisão numérica (16-bit, 8-bit)
    \item Limitação de bandas espectrais processadas simultaneamente
    \item Simulação de latências de memória e I/O
\end{itemize}

\subsection{Inserção de Artefatos Realísticos}

\subsubsection{Ruídos de Sensor}
Modelagem de ruídos típicos de sensores embarcados:
\begin{itemize}
    \item Ruído térmico dependente de temperatura
    \item Ruído de quantização
    \item Drift temporal de calibração
    \item Não-uniformidade de pixels
\end{itemize}

Modelo de ruído implementado:
\begin{equation}
I_{noisy}(i,j,\lambda) = I_{clean}(i,j,\lambda) + \eta_{thermal} + \eta_{quant} + \eta_{drift}(t)
\end{equation}

\subsubsection{Variações Ambientais}
Simulação de condições adversas de operação:
\begin{itemize}
    \item Variações de iluminação
    \item Efeitos atmosféricos variáveis
    \item Vibração e movimento de plataforma
    \item Oclusões parciais
\end{itemize}

\subsubsection{Limitações de Conectividade}
Simulação de restrições de comunicação:
\begin{itemize}
    \item Processamento local obrigatório
    \item Bandwidth limitado para transmissão
    \item Interrupções de conectividade
\end{itemize}

\section{Framework de Simulação Embarcada}\label{sec:framework_simulacao}

O framework desenvolvido permite validação controlada e reprodutível de algoritmos embarcados.

\subsection{Simulação via GHDL}

\subsubsection{Modelagem de Arquiteturas FPGA}
Desenvolvimento de modelos VHDL para componentes críticos:
\begin{itemize}
    \item Unidades de processamento espectral
    \item Controladores de memória otimizados
    \item Interfaces de comunicação
    \item Gerenciadores de energia
\end{itemize}

Exemplo de entidade VHDL para processamento espectral:
\begin{lstlisting}[language=VHDL]
entity spectral_processor is
    generic (
        BANDS : integer := 224;
        DATA_WIDTH : integer := 16;
        PARALLEL_UNITS : integer := 8
    );
    port (
        clk : in std_logic;
        rst : in std_logic;
        data_in : in std_logic_vector(DATA_WIDTH-1 downto 0);
        valid_in : in std_logic;
        data_out : out std_logic_vector(DATA_WIDTH-1 downto 0);
        valid_out : out std_logic
    );
end entity;
\end{lstlisting}

\subsubsection{Testbenches Abrangentes}
Desenvolvimento de testbenches que validam:
\begin{itemize}
    \item Correção funcional dos algoritmos
    \item Desempenho temporal
    \item Consumo de recursos
    \item Robustez a variações de entrada
\end{itemize}

\subsubsection{Análise de Timing e Power}
Ferramentas integradas para análise de:
\begin{itemize}
    \item Critical path timing
    \item Clock domain crossing
    \item Power consumption estimation
    \item Resource utilization
\end{itemize}

\subsection{Simulação de Plataformas Embarcadas}

\subsubsection{Modelos de VPU}
Simulação de características de Vision Processing Units:
\begin{itemize}
    \item Arquitetura de pipeline específica
    \item Hierarquia de memória otimizada
    \item Unidades de processamento especializado
\end{itemize}

\subsubsection{Modelos de GPU Embarcada}
Simulação de limitações de GPUs embarcadas:
\begin{itemize}
    \item Número reduzido de cores
    \item Limitações de memória
    \item Throttling térmico
    \item Gerenciamento de energia
\end{itemize}

\section{Métricas de Avaliação Embarcada}\label{sec:metricas_embarcada}

As métricas são organizadas para capturar aspectos específicos de sistemas embarcados:

\subsection{Métricas de Eficiência Energética}

\subsubsection{Consumo Instantâneo}
\begin{itemize}
    \item \textbf{Potência de Processamento}: Medição durante operação ativa
    \item \textbf{Potência de Standby}: Consumo em modo de espera
    \item \textbf{Picos de Consumo}: Análise de transientes energéticos
\end{itemize}

Métrica de eficiência energética:
\begin{equation}
E_{eff} = \frac{GOPS}{P_{avg}} \text{ [GOPS/W]}
\end{equation}
onde $GOPS$ representa operações por segundo e $P_{avg}$ a potência média.

\subsubsection{Energia Total por Operação}
\begin{itemize}
    \item \textbf{Energia por Pixel}: Custo energético de processamento
    \item \textbf{Energia por Classificação}: Custo de decisão completa
    \item \textbf{Overhead de Inicialização}: Custo de setup do sistema
\end{itemize}

\subsection{Métricas de Latência}

\subsubsection{Latência de Pipeline}
\begin{itemize}
    \item \textbf{Latência de Aquisição}: Tempo sensor-to-memory
    \item \textbf{Latência de Processamento}: Tempo de algoritmos
    \item \textbf{Latência de Decisão}: Tempo total sensor-to-output
\end{itemize}

\subsubsection{Jitter e Variabilidade}
\begin{itemize}
    \item \textbf{Jitter Temporal}: Variação na latência
    \item \textbf{Predictabilidade}: Consistência de timing
    \item \textbf{Worst-case Latency}: Cenário mais adverso
\end{itemize}

\subsection{Métricas de Qualidade}

\subsubsection{Precisão Sob Restrições}
\begin{itemize}
    \item \textbf{Acurácia vs. Energia}: Trade-off fundamental
    \item \textbf{Robustez a Ruído}: Tolerância a condições adversas
    \item \textbf{Estabilidade Temporal}: Consistência ao longo do tempo
\end{itemize}

\subsubsection{Graceful Degradation}
\begin{itemize}
    \item \textbf{Adaptação Dinâmica}: Ajuste a recursos disponíveis
    \item \textbf{Fallback Modes}: Operação com recursos limitados
    \item \textbf{Quality Scaling}: Ajuste de qualidade vs. recursos
\end{itemize}

\section{Protocolo de Validação}\label{sec:protocolo_validacao}

O protocolo estabelece procedimentos padronizados para validação reprodutível:

\subsection{Configuração de Ambiente}
\begin{itemize}
    \item \textbf{Calibração Inicial}: Setup controlado de hardware
    \item \textbf{Baseline Measurements}: Medições de referência
    \item \textbf{Environmental Monitoring}: Controle de condições
\end{itemize}

\subsection{Procedimentos de Teste}
\begin{itemize}
    \item \textbf{Warm-up Period}: Estabilização térmica
    \item \textbf{Multiple Runs}: Repetições para significância estatística
    \item \textbf{Statistical Analysis}: Tratamento rigoroso dos dados
    \item \textbf{Documentation}: Registro detalhado de condições
\end{itemize}

A próxima seção apresenta as estratégias específicas desenvolvidas para otimização energética e redução de latência baseadas nesta metodologia.

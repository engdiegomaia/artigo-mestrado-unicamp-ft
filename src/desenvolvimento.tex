%% Capítulo 3: Metodologia de Validação
%% Focado na Validação Conceitual e Metodológica da Etapa 1

\section{Visão Geral da Metodologia de Validação}

Esta pesquisa adota uma metodologia de \textbf{validação conceitual} estruturada em quatro fases para avaliar o potencial de integração de técnicas comprovadas em sistemas heterogêneos para processamento hiperespectral embarcado. O objetivo é estabelecer bases metodológicas sólidas através de análise sistemática, modelagem teórica e protótipos de prova de conceito.

A metodologia de validação baseia-se em três pilares fundamentais: \textbf{(1)} análise sistemática da literatura para catalogação de técnicas comprovadas, \textbf{(2)} modelagem e simulação para quantificação de trade-offs, e \textbf{(3)} protótipos conceituais para validação experimental dos conceitos mais promissores.

\section{Framework Arquitetural Conceitual}

\subsection{Modelo de Sistema Heterogêneo}

Para fins de validação conceitual, define-se um modelo de sistema heterogêneo com pipeline de três estágios especializados que servirá como base para simulações e análises:

\begin{enumerate}
\item \textbf{Estágio de Pré-processamento (FPGA)}: Correção radiométrica, seleção de bandas e compressive sensing
\item \textbf{Estágio de Processamento (GPU)}: Reconstrução de dados e extração de características
\item \textbf{Estágio de Classificação (CPU)}: Algoritmos de classificação e controle do sistema
\end{enumerate}

\subsection{Plataforma de Referência para Modelagem}

A modelagem teórica baseia-se em uma plataforma de referência representativa do estado da arte em sistemas embarcados:
\begin{itemize}
\item \textbf{Processamento}: ARM Cortex-A78 + NVIDIA Jetson Orin + Xilinx Zynq UltraScale+
\item \textbf{Memória}: 16GB LPDDR5 compartilhada
\item \textbf{Orçamento Energético}: 15W TDP total
\item \textbf{Throughput Alvo}: >25 fps para imagens 614×512×224 bandas
\end{itemize}

Esta configuração baseia-se em validações experimentais da literatura \cite{diaz2019, hwang2011} e representa o estado da arte atual em plataformas heterogêneas embarcadas.

\subsection{Estudo de Caso Industrial: SightLine Applications}

Para contextualizar a aplicação prática dos conceitos de computação heterogênea, analisa-se o caso da \textbf{SightLine Applications}, uma empresa líder no fornecimento de soluções de processamento de vídeo embarcado para aplicações de Inteligência, Vigilância e Reconhecimento (ISR), especialmente em Veículos Aéreos Não Tripulados (VANTs). As soluções da empresa precisam operar com restrições severas de Tamanho, Peso e Potência (SWaP), tornando a eficiência energética um requisito fundamental.

O produto de destaque da empresa, o processador de vídeo \textbf{4100-OEM}, é um exemplo claro de um sistema heterogêneo. Ele é baseado no System-on-Module (SOM) \textbf{Open-Q 8250CS} da Lantronix, que utiliza o SoC \textbf{Qualcomm QCS8250} \cite{lantronix2023openq}. A arquitetura deste SoC é intrinsecamente heterogênea, distribuindo as cargas de trabalho entre múltiplos núcleos de processamento especializados:

\begin{itemize}
    \item \textbf{CPU Octa-core Kryo™ 585}: Unidade baseada em ARM Cortex, responsável pelo sistema operacional, controle de fluxo, e tarefas de gerenciamento geral.
    \item \textbf{GPU Adreno™ 650}: Acelerador gráfico para processamento massivamente paralelo, ideal para tarefas como renderização, filtragem de imagem e transformações geométricas.
    \item \textbf{DSP Hexagon™}: Processador de Sinal Digital com extensões vetoriais, otimizado para algoritmos matemáticos de baixa latência, como estabilização de imagem e processamento de sinais de sensores.
    \item \textbf{NPU (Neural Processing Unit)}: Um motor de IA dedicado, capaz de executar até 15 trilhões de operações por segundo (TOPS), acelerando cargas de trabalho de \textit{deep learning} para detecção e classificação de objetos em tempo real.
    \item \textbf{ISP (Image Signal Processor) Spectra™ 480}: Unidade dedicada ao pipeline de processamento de imagem, lidando com tarefas como demosaicing, balanço de branco e redução de ruído diretamente do sensor.
\end{itemize}

A abordagem da SightLine, ao adotar um SoC heterogêneo como o QCS8250, permite que cada componente do pipeline de processamento de vídeo seja executado no núcleo mais eficiente para aquela tarefa específica. Isso resulta em um sistema de alta performance, capaz de processar múltiplos streams de vídeo em alta resolução com baixa latência e, crucialmente, dentro de um envelope de consumo energético extremamente restrito (tipicamente entre 5W e 15W). Este estudo de caso valida a premissa de que a computação heterogênea não é apenas uma construção teórica, but uma necessidade prática para a próxima geração de sistemas embarcados inteligentes.


\section{Fase 1: Análise Sistemática do Estado da Arte (2 meses)}

\subsection{Catalogação de Técnicas Comprovadas}

Esta fase concentra-se na análise sistemática da literatura para catalogar e caracterizar técnicas comprovadas de otimização para processamento hiperespectral embarcado:

\textbf{Objetivos Específicos}:
\begin{itemize}
\item Catalogação sistemática de técnicas de redução de dados (compressive sensing, seleção de bandas)
\item Análise de algoritmos de otimização energética e codesign HW/SW
\item Caracterização de implementações heterogêneas (CPU/GPU/FPGA) da literatura
\item Identificação de lacunas e oportunidades de integração
\end{itemize}

\textbf{Metodologia de Análise}:
\begin{itemize}
\item \textbf{Revisão Sistemática}: Análise de 20+ artigos científicos com critérios de seleção definidos
\item \textbf{Matriz de Caracterização}: Organização das técnicas por categoria (redução de dados, otimização energética, codesign)
\item \textbf{Análise Quantitativa}: Extração de métricas de performance, consumo e precisão da literatura
\item \textbf{Identificação de Lacunas}: Mapeamento de oportunidades para integração sistemática
\end{itemize}

\subsection{Estabelecimento de Baselines}

\textbf{Datasets de Validação}:
Baseados nas referências da literatura \cite{lou2024, ullah2020}:
\begin{itemize}
\item \textbf{AVIRIS Indian Pines}: Dataset padrão (145×145×220) para agricultura
\item \textbf{Pavia University}: Ambiente urbano (610×340×103) para robustez
\item \textbf{Salinas Valley}: Agricultura diversificada (512×217×224)
\end{itemize}

\textbf{Métricas de Avaliação}:
\begin{itemize}
\item \textbf{Performance}: Throughput (fps), latência (ms/pixel), utilização recursos (%)
\item \textbf{Energética}: Consumo total (W), eficiência energética (fps/W)
\item \textbf{Qualidade}: Precisão (%), recall (%), F1-score, kappa coefficient
\end{itemize}

\section{Fase 2: Modelagem e Simulação Conceitual (3 meses)}

\subsection{Modelagem Teórica das Técnicas}

Esta fase desenvolve modelos matemáticos e simulações para validar o potencial das técnicas identificadas na Fase 1:

\subsubsection{Modelagem de Técnicas de Redução de Dados}

\begin{itemize}
\item \textbf{Compressive Sensing}: Modelagem da redução de 50-70\% dos dados (Lim et al.) com análise de trade-offs precisão vs compressão
\item \textbf{Seleção EMCR}: Simulação da redução de 80\% das bandas mantendo 99.7\% de precisão (Martins et al.)
\item \textbf{Codesign HW/SW}: Modelagem do potencial de melhoria energética de 43.5x (Hwang et al.)
\end{itemize}

\textbf{Objetivo}: Quantificar através de simulações o potencial teórico de cada técnica e suas combinações.

\subsubsection{Simulação de Trade-offs}

Desenvolvimento de simuladores para análise quantitativa dos trade-offs:

\begin{itemize}
\item \textbf{Simulador de Consumo Energético}: Modelo baseado em medições da literatura para estimar consumo por módulo
\item \textbf{Simulador de Latência}: Análise de pipeline com diferentes configurações de paralelização
\item \textbf{Simulador de Precisão}: Avaliação do impacto das reduções de dados na qualidade final
\end{itemize}

\subsection{Análise de Sensibilidade}

\textbf{Parâmetros de Análise}:
\begin{itemize}
\item Taxa de compressão do compressive sensing (10-70\%)
\item Número de bandas selecionadas EMCR (20-100 bandas)
\item Particionamento de carga entre módulos (0-100\% por módulo)
\item Precisão de ponto flutuante (FP16, FP32, INT8)
\end{itemize}

\textbf{Métricas de Saída}:
\begin{itemize}
\item Consumo energético estimado por configuração
\item Latência teórica end-to-end
\item Precisão de classificação esperada
\item Throughput máximo alcançável
\end{itemize}

\section{Fase 3: Protótipos de Prova de Conceito (3 meses)}

\subsection{Implementação de Protótipos Simplificados}

Esta fase implementa protótipos simplificados das técnicas mais promissoras para validação experimental dos conceitos:

\subsubsection{Protótipo de Compressive Sensing}

\begin{itemize}
\item \textbf{Plataforma}: MATLAB/Python com bibliotecas otimizadas
\item \textbf{Objetivo}: Validar experimentalmente as reduções teóricas de dados
\item \textbf{Métricas}: Taxa de compressão vs precisão de reconstrução
\end{itemize}

\subsubsection{Protótipo de Seleção EMCR}

\begin{itemize}
\item \textbf{Plataforma}: Python com scikit-learn otimizado
\item \textbf{Objetivo}: Confirmar redução de processamento mantendo precisão
\item \textbf{Métricas}: Número de bandas vs precisão de classificação
\end{itemize}

\subsubsection{Protótipo de Pipeline Heterogêneo}

\begin{itemize}
\item \textbf{Plataforma}: Simulação em MATLAB Simulink
\item \textbf{Objetivo}: Avaliar coordenação entre módulos e balanceamento de carga
\item \textbf{Métricas}: Utilização de recursos vs throughput total
\end{itemize}

\subsection{Validação Experimental}

\textbf{Protocolo de Teste}:
\begin{enumerate}
\item Execução de cada protótipo nos datasets de referência
\item Medição das métricas de performance, consumo e qualidade
\item Comparação com baselines da literatura
\item Análise de variabilidade e robustez dos resultados
\end{enumerate}

\textbf{Critérios de Validação}:
\begin{itemize}
\item Confirmação das métricas reportadas na literatura
\item Identificação de fatores limitantes não reportados
\item Quantificação da variabilidade entre diferentes datasets
\end{itemize}

\section{Fase 4: Framework Arquitetural e Diretrizes (2 meses)}

\subsection{Consolidação dos Resultados}

Esta fase consolida os resultados das fases anteriores em um framework arquitetural conceitual para orientar a implementação futura:

\subsubsection{Framework de Decisão}

Desenvolvimento de um framework para seleção de técnicas baseado em:
\begin{itemize}
\item \textbf{Características da Aplicação}: Requisitos de latência, precisão e consumo
\item \textbf{Restrições da Plataforma}: Recursos disponíveis e orçamento energético
\item \textbf{Características dos Dados}: Resolução espacial/espectral e complexidade da cena
\end{itemize}

\subsubsection{Especificações Técnicas}

Definição de especificações técnicas para a implementação na Etapa 2:
\begin{itemize}
\item \textbf{Arquitetura do Sistema}: Configuração otimizada dos módulos heterogêneos
\item \textbf{Protocolos de Comunicação}: Interfaces entre módulos FPGA/GPU/CPU
\item \textbf{Algoritmos Adaptativos}: Estratégias de ajuste dinâmico de qualidade vs recursos
\item \textbf{Métricas de Monitoramento}: Indicadores para controle em tempo real
\end{itemize}

\subsection{Diretrizes para a Etapa 2}

\textbf{Recomendações Arquiteturais}:
\begin{itemize}
\item Configuração ótima de módulos baseada na análise de trade-offs
\item Estratégias de implementação prioritárias
\item Riscos identificados e estratégias de mitigação
\end{itemize}

\textbf{Plano de Implementação}:
\begin{itemize}
\item Sequência de desenvolvimento dos módulos
\item Milestones de validação intermediária
\item Critérios de sucesso quantitativos
\end{itemize}

\section{Cronograma de Execução}

\subsection{Visão Geral do Cronograma}

O desenvolvimento desta pesquisa está estruturado em um cronograma detalhado de 14 meses, iniciando em agosto de 2025 e culminando com a defesa em setembro de 2026. O cronograma foi organizado em quatro etapas principais, cada uma com objetivos específicos e entregáveis bem definidos.

A Figura \ref{fig:cronograma} apresenta uma visualização detalhada do cronograma de execução, incluindo as dependências entre tarefas, milestones críticos e o progresso atual do projeto. Esta visualização também está disponível no arquivo \texttt{cronograma\_mestrado\_gantt.html} e permite acompanhar o progresso em tempo real através de uma interface interativa desenvolvida com D3.js.

\begin{figure}[htbp]
\centering
\includegraphics[width=0.95\textwidth]{figuras/cronograma_mestrado_gantt.png}
\caption{Cronograma detalhado do projeto de mestrado (Agosto 2025 - Setembro 2026). O gráfico mostra as quatro etapas principais, 12 tarefas detalhadas e 4 milestones críticos. A linha vermelha pontilhada indica a data atual, permitindo acompanhar o progresso do projeto. As cores representam as diferentes etapas: verde (Fundamentação Teórica), azul (Desenvolvimento Experimental), laranja (Análise e Redação) e rosa (Finalização).}
\label{fig:cronograma}
\end{figure}

\subsection{Etapa 1: Fundamentação Teórica e Estado da Arte (Agosto 2025 - Janeiro 2026)}

\textbf{Duração}: 5 meses (Agosto 2025 - Janeiro 2026)

\textbf{Objetivos Principais}:
\begin{itemize}
\item Revisão sistemática da literatura sobre técnicas de otimização para processamento hiperespectral
\item Análise detalhada de técnicas comprovadas de redução de dados e otimização energética
\item Caracterização de sistemas heterogêneos e suas aplicações em processamento embarcado
\item Desenvolvimento do framework conceitual e metodologia de validação
\end{itemize}

\textbf{Tarefas Detalhadas}:
\begin{enumerate}
\item \textbf{Revisão Sistemática da Literatura} (1.5 meses): Análise de 20+ artigos científicos com catalogação sistemática de técnicas
\item \textbf{Análise de Técnicas de Otimização} (2 meses): Caracterização quantitativa de compressive sensing, seleção EMCR e codesign HW/SW
\item \textbf{Caracterização de Sistemas Heterogêneos} (1.5 meses): Análise de arquiteturas CPU+GPU+FPGA e seus trade-offs
\item \textbf{Framework Conceitual e Metodologia} (1.5 meses): Desenvolvimento da metodologia de validação e especificações técnicas
\end{enumerate}

\textbf{Entregáveis}:
\begin{itemize}
\item Relatório de revisão sistemática com matriz de técnicas catalogadas
\item Framework conceitual para sistemas heterogêneos
\item Metodologia detalhada de validação experimental
\item Baseline teórico para comparação de resultados
\end{itemize}

\subsection{Etapa 2: Desenvolvimento Experimental e Validação (Janeiro - Maio 2026)}

\textbf{Duração}: 5 meses (Janeiro - Maio 2026)

\textbf{Objetivos Principais}:
\begin{itemize}
\item Configuração do ambiente experimental heterogêneo
\item Implementação de protótipos de prova de conceito
\item Validação experimental das técnicas identificadas
\item Quantificação dos trade-offs entre performance, consumo e precisão
\end{itemize}

\textbf{Tarefas Detalhadas}:
\begin{enumerate}
\item \textbf{Configuração do Ambiente Experimental} (1 mês): Setup da plataforma heterogênea e ferramentas de desenvolvimento
\item \textbf{Implementação de Protótipos} (2.5 meses): Desenvolvimento de módulos FPGA, GPU e CPU com integração
\item \textbf{Validação e Testes Experimentais} (2.5 meses): Execução de experimentos e coleta de dados de performance
\end{enumerate}

\textbf{Entregáveis}:
\begin{itemize}
\item Plataforma experimental funcional
\item Protótipos implementados e validados
\item Dados experimentais de performance e consumo
\item Análise preliminar dos resultados
\end{itemize}

\subsection{Etapa 3: Análise de Resultados e Redação (Maio - Agosto 2026)}

\textbf{Duração}: 4 meses (Maio - Agosto 2026)

\textbf{Objetivos Principais}:
\begin{itemize}
\item Análise estatística detalhada dos resultados experimentais
\item Redação dos capítulos principais da dissertação
\item Discussão dos resultados e suas implicações
\item Formulação de conclusões e trabalhos futuros
\end{itemize}

\textbf{Tarefas Detalhadas}:
\begin{enumerate}
\item \textbf{Análise Estatística dos Resultados} (1 mês): Processamento estatístico e validação de significância
\item \textbf{Redação dos Capítulos Principais} (2.5 meses): Escrita dos capítulos de introdução, metodologia e resultados
\item \textbf{Discussão e Conclusões} (1.5 meses): Análise crítica dos resultados e formulação de conclusões
\end{enumerate}

\textbf{Entregáveis}:
\begin{itemize}
\item Análise estatística completa dos resultados
\item Capítulos da dissertação redigidos
\item Discussão crítica dos achados
\item Conclusões e diretrizes para trabalhos futuros
\end{itemize}

\subsection{Etapa 4: Finalização e Preparação para Defesa (Agosto - Setembro 2026)}

\textbf{Duração}: 1.5 meses (Agosto - Setembro 2026)

\textbf{Objetivos Principais}:
\begin{itemize}
\item Revisão e ajustes finais da dissertação
\item Preparação da apresentação de defesa
\item Simulação de defesa e ajustes finais
\item Submissão final e preparação para defesa
\end{itemize}

\textbf{Tarefas Detalhadas}:
\begin{enumerate}
\item \textbf{Revisão e Ajustes Finais} (1 mês): Correção de erros e melhorias na dissertação
\item \textbf{Preparação da Apresentação} (1 mês): Desenvolvimento dos slides e ensaios
\item \textbf{Simulação de Defesa e Ajustes} (1 mês): Prática da apresentação e refinamentos
\end{enumerate}

\textbf{Entregáveis}:
\begin{itemize}
\item Dissertação final revisada e formatada
\item Apresentação de defesa preparada
\item Material de apoio para a banca
\item Documentação completa do projeto
\end{itemize}

\subsection{Milestones Principais}

O cronograma inclui quatro milestones críticos que marcam pontos de verificação importantes:

\begin{enumerate}
\item \textbf{M1: Framework Conceitual Completo} (Janeiro 2026): Conclusão da fundamentação teórica e metodologia
\item \textbf{M2: Protótipos Validados} (Maio 2026): Validação experimental das técnicas propostas
\item \textbf{M3: Dissertação Completa} (Agosto 2026): Documento final redigido e revisado
\item \textbf{M4: Defesa} (Setembro 2026): Apresentação e defesa da dissertação
\end{enumerate}

\subsection{Controle e Acompanhamento}

\textbf{Monitoramento Semanal}:
\begin{itemize}
\item Reuniões de acompanhamento com orientador
\item Atualização do progresso das tarefas
\item Identificação de riscos e ajustes no cronograma
\end{itemize}

\textbf{Revisões Mensais}:
\begin{itemize}
\item Avaliação do progresso geral do projeto
\item Ajustes no cronograma conforme necessário
\item Validação da qualidade dos entregáveis
\end{itemize}

\textbf{Contingências}:
\begin{itemize}
\item Buffer de tempo de 2 semanas em cada etapa para imprevistos
\item Plano alternativo para atrasos em tarefas críticas
\item Flexibilidade na sequência de algumas tarefas paralelas
\end{itemize}

\section{Metodologia de Análise dos Resultados}

\subsection{Análise Estatística}

\textbf{Testes Estatísticos}:
\begin{itemize}
\item ANOVA para comparação entre diferentes configurações
\item Teste t-student para validação de significância das melhorias
\item Análise de correlação entre métricas de trade-off
\end{itemize}

\textbf{Validação de Robustez}:
\begin{itemize}
\item Análise de sensibilidade a variações de parâmetros
\item Teste com diferentes datasets para generalização
\item Avaliação de estabilidade temporal dos resultados
\end{itemize}

\subsection{Comparação com Estado da Arte}

\textbf{Benchmarks de Referência}:
\begin{itemize}
\item Comparação com implementações CPU convencionais
\item Análise relativa às melhores soluções GPU/FPGA da literatura
\item Avaliação do potencial de melhoria teórico vs prático
\end{itemize}

\textbf{Métricas de Comparação}:
\begin{itemize}
\item Fator de melhoria energética (speedup energético)
\item Redução percentual de latência
\item Manutenção/melhoria da precisão
\item Viabilidade de implementação prática
\end{itemize}
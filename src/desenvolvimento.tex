% Capítulo de Metodologia - Avaliação de Processadores para Processamento Hiperespectral em Tempo Real
\chapter{Metodologia}\label{chp:metodologia}

Este capítulo apresenta a metodologia desenvolvida para avaliar o desempenho de diferentes arquiteturas de processamento (CPU, GPU e FPGA) no contexto de processamento de imagens hiperespectrais em tempo real. A metodologia proposta estabelece uma abordagem sistemática para análise comparativa de desempenho, incluindo simulações detalhadas e considerações sobre arquiteturas híbridas como uma extensão do estudo.

\section{Visão Geral da Metodologia de Avaliação}\label{sec:visao_geral}

A metodologia de avaliação baseia-se na premissa de que diferentes arquiteturas de processamento possuem características distintas que podem ser mais ou menos adequadas para diferentes tipos de operações hiperespectrais em tempo real.

% Figura da metodologia será incluída posteriormente
% \begin{figure}[!htb]
% \centering
% \includegraphics[width=0.9\textwidth]{metodologia_avaliacao.png}
% \caption[Metodologia de Avaliação]{Metodologia proposta para avaliação comparativa de processadores em processamento hiperespectral.}
% \label{fig:metodologia_avaliacao}
% \end{figure}

A metodologia é estruturada em cinco componentes principais:

\begin{enumerate}
    \item \textbf{Caracterização de Cargas de Trabalho}: Análise e classificação das operações hiperespectrais
    \item \textbf{Implementação por Arquitetura}: Desenvolvimento otimizado para cada processador
    \item \textbf{Framework de Simulação}: Ambiente controlado para testes e medições
    \item \textbf{Métricas de Avaliação}: Indicadores objetivos de desempenho
    \item \textbf{Análise Comparativa}: Avaliação sistemática dos resultados
\end{enumerate}

\section{Caracterização de Cargas de Trabalho}\label{sec:caracterizacao}

A caracterização das cargas de trabalho é fundamental para uma avaliação justa e objetiva. As operações são classificadas conforme a \Tabela{tab:caracterizacao}.

\begin{table}[!htp]
\caption[Caracterização de Operações]{Caracterização das operações de processamento hiperespectral.}
\label{tab:caracterizacao}
\begin{center}
\begin{tabular}{|p{3cm}|p{3cm}|p{3cm}|p{3cm}|}
\hline
\textbf{Operação} & \textbf{Complexidade} & \textbf{Paralelismo} & \textbf{Requisitos de Memória} \\
\hline
Correção Radiométrica & Baixa & Alto & Baixo \\
\hline
Filtros Espaciais & Média & Alto & Médio \\
\hline
Redução Dimensional & Alta & Médio & Alto \\
\hline
Classificação & Alta & Variável & Alto \\
\hline
\end{tabular}
\end{center}
\end{table}

\section{Implementação por Arquitetura}\label{sec:implementacao}

\subsection{Implementação CPU}
A implementação em CPU serve como baseline e utiliza otimizações modernas:

\begin{itemize}
    \item \textbf{Vetorização}: Uso de instruções SIMD (AVX-512)
    \item \textbf{Multi-threading}: Paralelização com OpenMP
    \item \textbf{Otimização de Cache}: Técnicas de tiling e loop fusion
    \item \textbf{Algoritmos Otimizados}: Implementações eficientes de operações críticas
\end{itemize}

\subsection{Implementação GPU}
A implementação em GPU explora paralelismo massivo via CUDA:

\begin{itemize}
    \item \textbf{Kernel Optimization}: Maximização de ocupância e coalescência de memória
    \item \textbf{Memory Management}: Uso eficiente de memória compartilhada e textura
    \item \textbf{Stream Processing}: Processamento assíncrono e overlapping
    \item \textbf{Warp-level Programming}: Otimizações específicas para warps
\end{itemize}

\subsection{Implementação FPGA}
A implementação em FPGA utiliza GHDL para simulação de arquiteturas customizadas:

\begin{itemize}
    \item \textbf{Pipeline Customizado}: Otimização para operações específicas
    \item \textbf{Paralelismo Fino}: Unidades de processamento especializadas
    \item \textbf{Memória Local}: Buffers e caches otimizados
    \item \textbf{Precisão Adaptativa}: Otimização de representação numérica
\end{itemize}

\section{Framework de Simulação}\label{sec:simulacao}

O framework de simulação permite avaliação controlada e reprodutível:

\begin{itemize}
    \item \textbf{Geração de Dados}: Datasets sintéticos e reais
    \item \textbf{Controle de Parâmetros}: Configuração de cenários
    \item \textbf{Instrumentação}: Coleta de métricas
    \item \textbf{Validação}: Verificação de resultados
\end{itemize}

\subsection{Ambiente de Simulação}
O ambiente inclui ferramentas específicas para cada arquitetura:

\begin{itemize}
    \item \textbf{CPU}: Profilers e contadores de hardware
    \item \textbf{GPU}: NVIDIA NSight e CUDA Profiler
    \item \textbf{FPGA}: GHDL e ferramentas de análise temporal
\end{itemize}

\section{Métricas de Avaliação}\label{sec:metricas}

As métricas são organizadas em quatro categorias principais:

\begin{enumerate}
    \item \textbf{Desempenho}:
    \begin{itemize}
        \item Throughput (pixels/segundo)
        \item Latência (ms)
        \item Utilização de recursos
    \end{itemize}
    
    \item \textbf{Eficiência Energética}:
    \begin{itemize}
        \item Consumo de energia (W)
        \item Eficiência energética (GOPS/W)
        \item Perfil térmico
    \end{itemize}
    
    \item \textbf{Qualidade}:
    \begin{itemize}
        \item Precisão numérica
        \item SNR
        \item Estabilidade temporal
    \end{itemize}
    
    \item \textbf{Custo-benefício}:
    \begin{itemize}
        \item Custo de implementação
        \item Complexidade de desenvolvimento
        \item Flexibilidade
    \end{itemize}
\end{enumerate}

\section{Análise de Arquiteturas Híbridas}\label{sec:hibridas}

Como extensão da análise individual, investigamos o potencial de arquiteturas híbridas:

\subsection{Cenários de Integração}
Avaliamos diferentes combinações de processadores:

\begin{itemize}
    \item \textbf{CPU-GPU}: Processamento heterogêneo tradicional
    \item \textbf{CPU-FPGA}: Aceleração de baixa latência
    \item \textbf{FPGA-GPU}: Pipeline especializado
\end{itemize}

\subsection{Estratégias de Particionamento}
Para arquiteturas híbridas, consideramos:

\begin{itemize}
    \item \textbf{Divisão por Operação}: Alocação baseada em características
    \item \textbf{Pipeline Temporal}: Processamento em estágios
    \item \textbf{Paralelismo Espacial}: Divisão por regiões de dados
\end{itemize}

\subsection{Análise de Trade-offs}
Avaliação de compromissos em integrações híbridas:

\begin{itemize}
    \item \textbf{Overhead de Comunicação}: Latência e bandwidth
    \item \textbf{Complexidade de Controle}: Sincronização e scheduling
    \item \textbf{Custo Total}: Hardware e desenvolvimento
    \item \textbf{Benefícios Potenciais}: Ganhos em cenários específicos
\end{itemize}

\section{Metodologia Experimental}\label{sec:metodologia_exp}

A validação experimental segue um protocolo rigoroso:

\subsection{Datasets de Teste}
Utilizamos três tipos de datasets:

\begin{itemize}
    \item \textbf{Sintéticos}: Controle total de parâmetros
    \item \textbf{Benchmark}: Datasets padrão da literatura
    \item \textbf{Reais}: Aplicações práticas específicas
\end{itemize}

\subsection{Cenários de Teste}
Avaliamos diferentes condições operacionais:

\begin{itemize}
    \item \textbf{Carga Variável}: Diferentes volumes de dados
    \item \textbf{Complexidade}: Variação de parâmetros algorítmicos
    \item \textbf{Restrições}: Limitações de recursos e tempo
\end{itemize}

\subsection{Protocolo de Validação}
Estabelecemos procedimentos padronizados:

\begin{itemize}
    \item \textbf{Calibração}: Setup inicial controlado
    \item \textbf{Repetibilidade}: Múltiplas execuções
    \item \textbf{Análise Estatística}: Tratamento rigoroso dos dados
    \item \textbf{Documentação}: Registro detalhado de condições
\end{itemize}

A próxima seção apresenta os resultados obtidos através desta metodologia, permitindo uma análise comparativa objetiva das diferentes arquiteturas de processamento no contexto de processamento hiperespectral em tempo real.

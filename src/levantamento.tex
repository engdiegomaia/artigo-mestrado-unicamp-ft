% Capítulo de Levantamento Bibliográfico / Trabalhos Relacionados
\chapter{Levantamento Bibliográfico}\label{chp:levantamento}

Este capítulo apresenta uma revisão sistemática da literatura sobre processamento hiperespectral embarcado com foco em estratégias para redução de consumo energético e latência. A revisão abrange desde os conceitos fundamentais do processamento hiperespectral em sistemas embarcados até as técnicas mais avançadas de otimização para aplicações práticas em agricultura de precisão, monitoramento ambiental e sistemas de vigilância.

\section{Processamento Hiperespectral Embarcado}\label{sec:hiperespectral_embarcado}

\subsection{Conceitos Fundamentais}
O processamento hiperespectral embarcado representa uma evolução necessária das técnicas tradicionais de sensoriamento remoto para atender demandas de aplicações de campo em tempo real. Diferentemente dos sistemas computacionais convencionais, as plataformas embarcadas operam sob restrições severas de energia, memória e capacidade de processamento, exigindo abordagens especializadas para o tratamento de dados hiperespectrais.

Cada pixel hiperespectral contém informações de centenas de bandas espectrais, resultando em volumes de dados que tradicionalmente requerem processamento em nuvem ou estações de trabalho. No contexto embarcado, esta característica fundamental cria desafios únicos para implementação de algoritmos eficientes que mantenham precisão adequada enquanto operam dentro das limitações de hardware \cite{Lou2024}.

\subsection{Arquiteturas de Hardware Embarcado}
O desenvolvimento de hardware especializado para processamento hiperespectral tem evoluído rapidamente, com várias arquiteturas emergindo como candidatas viáveis para implementações embarcadas:

\subsubsection{FPGAs (Field-Programmable Gate Arrays)}
FPGAs oferecem vantagens únicas para processamento hiperespectral embarcado devido à sua capacidade de customização e eficiência energética. A programabilidade do hardware permite otimizações específicas para operações hiperespectrais, incluindo pipelines customizados, paralelização fina e precisão aritmética adaptativa.

\subsubsection{VPUs (Vision Processing Units)}
Unidades de processamento especializado como Intel Movidius representam uma nova categoria de hardware otimizado para visão computacional embarcada. Estas unidades oferecem alto desempenho em operações de convoluções e outras operações comuns em processamento de imagens, com consumo energético significativamente reduzido comparado a GPUs tradicionais.

\subsubsection{GPUs Embarcadas}
Plataformas como NVIDIA Jetson series trazem capacidades de processamento paralelo massivo para ambientes embarcados, mantendo perfis de consumo apropriados para aplicações de campo. A programabilidade via CUDA permite implementações eficientes de algoritmos hiperespectrais complexos.

\section{Estratégias de Redução de Consumo Energético}\label{sec:reducao_consumo}

\subsection{Otimizações Algorítmicas}
A redução do consumo energético em processamento hiperespectral pode ser abordada através de múltiplas estratégias algorítmicas que mantêm a qualidade dos resultados:

\subsubsection{Compressão Adaptativa}
Técnicas de compressão específicas para dados hiperespectrais permitem redução significativa na quantidade de dados processados, resultando em menor consumo energético. Algoritmos adaptativos que ajustam parâmetros de compressão baseados nas características espectrais locais demonstram eficácia particular em cenários embarcados.

\subsubsection{Redução de Dimensionalidade Otimizada}
Implementações eficientes de PCA (Principal Component Analysis) e outras técnicas de redução dimensional específicas para hardware embarcado podem reduzir drasticamente a carga computacional. Abordagens que combinam redução espectral e espacial mostram-se promissoras para aplicações em tempo real.

\subsubsection{Processamento Hierárquico}
Estratégias que implementam processamento em múltiplas resoluções, começando com análises grosseiras e refinando apenas regiões de interesse, podem resultar em economias substanciais de energia mantendo precisão adequada para a aplicação.

\subsection{Otimizações de Hardware}
A eficiência energética também pode ser melhorada através de otimizações específicas ao nível de hardware:

\subsubsection{Gerenciamento Dinâmico de Energia}
Técnicas que ajustam dinamicamente frequências de operação e voltagens baseadas na carga computacional atual permitem redução significativa no consumo total. Clock gating e power gating seletivos podem ser aplicados a componentes não utilizados momentaneamente.

\subsubsection{Precisão Aritmética Adaptativa}
A utilização de representações numéricas de precisão reduzida (fixed-point, half-precision) para operações que não requerem alta precisão pode resultar em economias substanciais de energia sem comprometer significativamente a qualidade dos resultados.

\section{Técnicas de Redução de Latência}\label{sec:reducao_latencia}

\subsection{Paralelização Eficiente}
A natureza dos dados hiperespectrais oferece múltiplas oportunidades para paralelização que podem ser exploradas para redução de latência:

\subsubsection{Paralelização Espectral}
Processamento simultâneo de múltiplas bandas espectrais permite aproveitamento da estrutura natural dos dados hiperespectrais. Implementações que dividem o cubo espectral em sub-blocos processados em paralelo mostram eficácia particular em FPGAs.

\subsubsection{Paralelização Espacial}
Divisão da imagem em tiles processados simultaneamente permite escalabilidade com o número de unidades de processamento disponíveis. Técnicas de overlapping inteligente entre tiles minimizam artefatos de borda sem impacto significativo na latência.

\subsubsection{Pipeline de Processamento}
Implementação de pipelines que permitem processamento simultâneo de diferentes estágios algorítmicos pode reduzir drasticamente a latência total. Balanceamento cuidadoso das etapas do pipeline é crítico para maximizar throughput.

\subsection{Otimizações de Memória}
O gerenciamento eficiente de memória é crucial para redução de latência em sistemas embarcados:

\subsubsection{Hierarquia de Cache Otimizada}
Implementação de caches especializados que exploram padrões de acesso específicos de algoritmos hiperespectrais pode reduzir significativamente latências de acesso à memória. Cache blocking adaptativo mostra-se particularmente eficaz.

\subsubsection{Streaming de Dados}
Técnicas que permitem processamento de dados conforme são adquiridos, sem necessidade de armazenamento completo do cubo hiperespectral, são essenciais para aplicações de ultra-baixa latência.

\section{Aplicações Práticas Embarcadas}\label{sec:aplicacoes_praticas}

\subsection{Agricultura de Precisão}
O processamento hiperespectral embarcado tem encontrado aplicações crescentes em agricultura de precisão, onde a necessidade de respostas em tempo real é crítica:

\subsubsection{Detecção de Estresse em Culturas}
Sistemas embarcados em drones agrícolas permitem detecção imediata de estresses nutricionais, hídricos ou patológicos. Algoritmos otimizados para hardware embarcado conseguem identificar assinaturas espectrais características de diferentes tipos de estresse, permitindo intervenções precisas e oportunas \cite{Shin2024}.

\subsubsection{Monitoramento de Crescimento}
Análise contínua do desenvolvimento das culturas através de índices espectrais calculados em tempo real permite otimização do manejo agrícola. Implementações embarcadas permitem coleta de dados de alta frequência temporal sem dependência de conectividade externa.

\subsubsection{Aplicação Variável de Insumos}
Sistemas que combinam aquisição hiperespectral embarcada com aplicação automatizada de fertilizantes ou pesticidas representam a fronteira atual da agricultura de precisão. A latência ultra-baixa é crítica para sincronização entre detecção e aplicação.

\subsection{Monitoramento Ambiental}
Aplicações ambientais embarcadas requerem operação autônoma em condições adversas por períodos prolongados:

\subsubsection{Detecção Precoce de Queimadas}
Sistemas embarcados para detecção automática de focos de incêndio utilizam assinaturas espectrais específicas de combustão inicial. A redução de latência é crítica para permitir respostas rápidas a emergências. Implementações que combinam processamento hiperespectral com análise térmica mostram eficácia particular.

\subsubsection{Monitoramento de Qualidade da Água}
Sensores hiperespectrais embarcados para monitoramento contínuo de corpos d'água requerem operação de baixo consumo por longos períodos. Algoritmos adaptativos que ajustam frequência de amostragem baseados na qualidade detectada podem otimizar autonomia energética.

\subsubsection{Detecção de Mudanças na Cobertura Vegetal}
Sistemas embarcados para monitoramento de desmatamento ou degradação ambiental operam tipicamente em locais remotos com energia limitada. Técnicas de change detection otimizadas para processamento embarcado permitem detecção automática de alterações significativas na paisagem.

\subsection{Sistemas de Vigilância e Segurança}
Aplicações de segurança demandam combinação de baixa latência com operação discreta:

\subsubsection{Reconhecimento de Materiais}
Identificação automática de substâncias específicas através de assinaturas espectrais únicas requer algoritmos embarcados capazes de operar com libraries espectrais extensas. Técnicas de compressão e indexação otimizadas são essenciais para viabilidade prática.

\subsubsection{Detecção de Objetos Ocultos}
Sistemas hiperespectrais embarcados podem identificar materiais camuflados ou ocultos explorando diferenças espectrais não visíveis ao olho humano. A combinação de processamento espectral e espacial em hardware embarcado permite detecção em tempo real.

\subsubsection{Monitoramento Perimetral}
Sistemas de vigilância embarcados para grandes áreas requerem operação autônoma com resposta imediata a intrusões. Algoritmos que combinam detecção de movimento com análise espectral para classificação de alvos mostram eficácia em reduzir falsos positivos.

\section{Caracterização e Seleção de Datasets}\label{sec:datasets}

\subsection{Metodologias de Caracterização}
A validação efetiva de sistemas hiperespectrais embarcados requer datasets que representem fielmente as condições operacionais reais:

\subsubsection{Critérios de Representatividade}
Datasets devem ser avaliados quanto à correspondência com cenários de aplicação específicos, incluindo condições ambientais, características dos alvos e variabilidade temporal. Métricas quantitativas de representatividade permitem seleção objetiva de dados de teste apropriados.

\subsubsection{Diversidade de Condições}
Cobertura adequada de diferentes condições operacionais é essencial para validação robusta. Datasets devem incluir variações de iluminação, condições atmosféricas, sazonalidade e características dos alvos para garantir generalização dos algoritmos desenvolvidos.

\subsubsection{Metadados Auxiliares}
Informações complementares sobre condições de aquisição, calibração de sensores e verdade terrestre são críticas para interpretação correta dos resultados. Protocolos padronizados para documentação de metadados facilitam comparação entre estudos.

\subsection{Preparação para Simulação de Operações Reais}
Adaptação de datasets existentes para simular condições operacionais embarcadas requer técnicas específicas:

\subsubsection{Simulação de Aquisição Streaming}
Conversão de datasets tradicionais para formato de streaming que simule aquisição em tempo real permite validação de algoritmos embarcados sob condições realísticas de processamento contínuo.

\subsubsection{Inserção de Ruídos e Artefatos}
Adição controlada de ruídos, variações de calibração e artefatos típicos de sistemas embarcados permite validação da robustez dos algoritmos sob condições não-ideais.

\subsubsection{Variação de Resolução}
Simulação de diferentes resoluções espectrais e espaciais permite avaliação da degradação graceful dos algoritmos quando operando com sensores de menor capacidade típicos de sistemas embarcados.

\section{Tecnologias de Simulação e Validação}\label{sec:simulacao_validacao}

\subsection{GHDL para Simulação de FPGAs}
O GHDL (VHDL simulator and analyzer) representa uma ferramenta fundamental para desenvolvimento e validação de implementações em FPGA:

\subsubsection{Modelagem de Precisão}
GHDL permite modelagem precisa do comportamento temporal e energético de designs VHDL, facilitando otimização antes da síntese em hardware real. Simulações detalhadas podem identificar gargalos e oportunidades de otimização.

\subsubsection{Validação Funcional}
Verificação completa da correção funcional de implementações VHDL através de testbenches abrangentes reduz riscos de erros em hardware final. Técnicas de verificação formal podem garantir correção de propriedades críticas.

\subsubsection{Estimativa de Recursos}
Análise detalhada de utilização de recursos (LUTs, DSPs, memória) permite otimização de designs para plataformas específicas antes da síntese física.

\subsection{Ferramentas de Análise Energética}
Medição precisa de consumo energético é essencial para validação de estratégias de otimização:

\subsubsection{Profiling Energético}
Ferramentas especializadas permitem medição detalhada do consumo energético de diferentes componentes e operações, facilitando identificação de oportunidades de otimização.

\subsubsection{Modelagem de Consumo}
Modelos analíticos de consumo baseados em características dos algoritmos e hardware permitem estimativa rápida sem necessidade de implementação física completa.

\section{Trabalhos Relacionados}\label{sec:trabalhos_relacionados}

\subsection{Processamento Hiperespectral Embarcado}
\cite{Lim2022} investigou a viabilidade de sistemas embarcados em tempo real para imageamento hiperespectral compressivo, analisando requisitos computacionais, uso de memória e largura de banda. O trabalho propôs otimizações específicas para alcançar processamento em tempo real, focando em técnicas de compressive sensing adaptadas para hardware embarcado.

\subsection{Otimizações para Agricultura de Precisão}
\cite{Shin2024} apresentou um estudo focado em métodos robustos de correção radiométrica e geométrica para imageamento hiperespectral baseado em drones em aplicações agrícolas. O trabalho demonstrou a importância de técnicas de pré-processamento otimizadas para sistemas embarcados operando em condições de campo.

\subsection{Eficiência Energética em Visão Computacional}
Estudos recentes em visão computacional embarcada têm explorado técnicas de quantização, pruning e destilação de conhecimento para redução de consumo energético. Estas técnicas mostram-se promissoras para adaptação ao processamento hiperespectral embarcado.

\subsection{Arquiteturas Especializadas}
O desenvolvimento de processadores especializados para visão computacional, incluindo TPUs (Tensor Processing Units) e NPUs (Neural Processing Units), oferece oportunidades para aceleração eficiente de algoritmos hiperespectrais específicos.

\subsection{Lacunas Identificadas na Literatura}
A revisão da literatura identifica várias lacunas importantes para processamento hiperespectral embarcado:

\begin{enumerate}
    \item \textbf{Metodologias de Caracterização de Datasets}: Falta de protocolos padronizados para caracterização de datasets que simulem fielmente operações embarcadas reais
    \item \textbf{Métricas de Eficiência Embarcada}: Necessidade de métricas específicas que considerem simultaneamente precisão, consumo energético e latência em aplicações práticas
    \item \textbf{Validação em Condições Reais}: Escassez de estudos validando metodologias em condições operacionais reais de campo por períodos prolongados
    \item \textbf{Frameworks Integrados}: Ausência de frameworks que integrem eficientemente caracterização de dados, otimização algorítmica e implementação embarcada
    \item \textbf{Transferência de Tecnologia}: Limitado número de estudos focando na implementação prática e transferência para produtos comerciais viáveis
\end{enumerate}

\section{Síntese do Levantamento}\label{sec:sintese}

Esta revisão da literatura estabelece o contexto científico e tecnológico para o desenvolvimento desta dissertação focada em estratégias para redução de consumo e latência no processamento hiperespectral embarcado:

\subsection{Estado da Arte}
\begin{itemize}
    \item O processamento hiperespectral embarcado representa uma área emergente com grande potencial para aplicações práticas críticas
    \item Técnicas de otimização energética e redução de latência específicas para dados hiperespectrais demonstram viabilidade técnica
    \item Metodologias de caracterização de datasets para simulação de operações reais são fundamentais para validação efetiva
    \item GHDL e outras ferramentas de simulação permitem desenvolvimento e validação de implementações embarcadas antes da síntese física
\end{itemize}

\subsection{Desafios Técnicos}
\begin{itemize}
    \item Balanceamento entre precisão, consumo energético e latência em aplicações críticas
    \item Desenvolvimento de algoritmos robustos para operação em condições adversas de campo
    \item Validação adequada de sistemas embarcados em cenários operacionais reais
    \item Integração eficiente de múltiplas técnicas de otimização sem comprometer funcionalidade
\end{itemize}

\subsection{Oportunidades de Pesquisa}
Esta dissertação visa contribuir para o preenchimento das lacunas identificadas através de:
\begin{itemize}
    \item Desenvolvimento de metodologia integrada para caracterização de datasets e validação embarcada
    \item Implementação e validação de estratégias de otimização específicas para aplicações práticas
    \item Estabelecimento de diretrizes para deployment de sistemas hiperespectrais embarcados
    \item Demonstração de viabilidade através de protótipos funcionais validados em condições reais
\end{itemize}

Os capítulos subsequentes detalham a metodologia desenvolvida e os resultados obtidos, contribuindo para o avanço do conhecimento nesta área estratégica para múltiplas aplicações práticas.

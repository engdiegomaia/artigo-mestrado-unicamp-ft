% Capítulo de Levantamento Bibliográfico / Trabalhos Relacionados
\chapter{Levantamento Bibliográfico}\label{chp:levantamento}

Este capítulo apresenta uma revisão sistemática da literatura sobre classificação de uso e cobertura da terra (LULC) utilizando imagens hiperespectrais, com foco especial em aplicações com drones/VANTs. A revisão abrange desde os conceitos fundamentais do sensoriamento remoto hiperespectral até as técnicas mais avançadas de processamento e classificação baseadas em deep learning.

\section{Sensoriamento Remoto Hiperespectral}\label{sec:hiperespectral}

\subsection{Conceitos Fundamentais}
O sensoriamento remoto hiperespectral representa uma evolução significativa em relação às técnicas multiespectrais tradicionais. Enquanto imagens multiespectrais capturam informações em poucas bandas espectrais discretas (tipicamente 3-10 bandas), as imagens hiperespectrais adquirem dados em centenas de bandas espectrais contíguas e estreitas, geralmente variando de 5 a 20 nanômetros de largura \cite{Lou2024}.

Esta rica informação espectral permite uma caracterização muito mais precisa dos materiais e coberturas terrestres, possibilitando a distinção entre objetos que apresentam assinaturas espectrais similares nas poucas bandas de sensores multiespectrais convencionais. O resultado é uma estrutura de dados tridimensional chamada "cubo hiperespectral" ou "cubo de dados", onde as duas primeiras dimensões representam informações espaciais e a terceira dimensão contém informações espectrais \cite{Lou2024}.

\subsection{Evolução Tecnológica}
O desenvolvimento da tecnologia de imageamento hiperespectral teve início com espectrômetros imageadores aerotransportados, evoluindo para sistemas como o Advanced Visible and Infrared Imaging Spectrometer (AVIRIS), que se tornou uma referência na geração de datasets hiperespectrais acadêmicos \cite{Lou2024}. 

Nos últimos anos, a miniaturização dos sensores hiperespectrais e o desenvolvimento de plataformas não tripuladas revolucionaram o campo, tornando essa tecnologia mais acessível e permitindo aplicações em escalas locais e regionais com alta resolução temporal e espacial.

\subsection{Aplicações em Monitoramento Terrestre}
As imagens hiperespectrais têm encontrado aplicações em diversos campos científicos e comerciais:

\begin{itemize}
    \item \textbf{Agricultura de Precisão}: Monitoramento da saúde das culturas, detecção precoce de estresses hídricos e nutricionais, identificação de pragas e doenças \cite{Shin2024}
    \item \textbf{Monitoramento Florestal}: Classificação de espécies vegetais, avaliação de biomassa, detecção de mudanças na cobertura vegetal
    \item \textbf{Geologia}: Identificação e mapeamento de minerais, prospecção mineral, estudos geológicos estruturais
    \item \textbf{Recursos Hídricos}: Monitoramento da qualidade da água, detecção de poluição aquática, mapeamento de corpos d'água
    \item \textbf{Monitoramento Ambiental}: Avaliação de impactos ambientais, detecção de mudanças no uso da terra, conservação da biodiversidade
\end{itemize}

\section{Classificação LULC com Imagens Hiperespectrais}\label{sec:classificacao_lulc}

\subsection{Definição e Importância da Classificação LULC}
A classificação de uso e cobertura da terra (Land Use/Land Cover - LULC) constitui uma das aplicações mais importantes do sensoriamento remoto hiperespectral. Segundo \cite{Lou2024}, LULC pode ser categorizada em dois domínios principais:

\begin{itemize}
    \item \textbf{Cobertura da Terra (Land Cover)}: Refere-se aos fatores naturais e antrópicos existentes na superfície terrestre, incluindo florestas, pastagens, terras agrícolas, solo exposto, corpos d'água e áreas construídas.
    \item \textbf{Uso da Terra (Land Use)}: Representa o processo de transformação do ecossistema natural em sistemas artificiais, influenciado por fatores naturais, econômicos e sociais, incluindo áreas residenciais, comerciais, industriais e destinadas ao transporte.
\end{itemize}

A classificação precisa de LULC é fundamental para o planejamento territorial, gestão de recursos naturais, monitoramento de mudanças ambientais e formulação de políticas públicas de desenvolvimento sustentável.

\subsection{Desafios na Classificação LULC Hiperespectral}
O processamento de imagens hiperespectrais para classificação LULC apresenta desafios únicos devido às características intrínsecas dos dados:

\subsubsection{Maldição da Dimensionalidade}
A alta dimensionalidade dos dados hiperespectrais (centenas de bandas espectrais) pode levar ao fenômeno conhecido como "maldição da dimensionalidade" (curse of dimensionality), onde o desempenho dos algoritmos de classificação pode degradar devido à escassez relativa de amostras de treinamento em espaços de alta dimensionalidade \cite{Lou2024}.

\subsubsection{Redundância Espectral}
Muitas bandas espectrais adjacentes apresentam alta correlação, introduzindo redundância nos dados que pode afetar negativamente o desempenho dos classificadores e aumentar o custo computacional \cite{Lou2024}.

\subsubsection{Variabilidade Espectral}
A mesma classe de cobertura pode apresentar variações espectrais significativas devido a fatores como estado fenológico da vegetação, condições de iluminação, propriedades do solo e efeitos atmosféricos \cite{Lou2024}.

\subsection{Evolução dos Métodos de Classificação}
A literatura apresenta uma evolução clara nos métodos de classificação LULC hiperespectral, que podem ser organizados em três gerações principais \cite{Lou2024}:

\subsubsection{Primeira Geração: Métodos Estatísticos Tradicionais}
Os primeiros métodos incluíam classificadores como:
\begin{itemize}
    \item \textbf{Máxima Verossimilhança (Maximum Likelihood)}: Baseado na assumptiva de distribuições gaussianas para as classes
    \item \textbf{Distância Mínima}: Classificação baseada na distância euclidiana até centroides das classes
    \item \textbf{Paralelepípedo}: Definição de limites espectrais para cada banda
\end{itemize}

\subsubsection{Segunda Geração: Métodos de Machine Learning}
O desenvolvimento de técnicas de aprendizado de máquina trouxe melhorias significativas:
\begin{itemize}
    \item \textbf{Support Vector Machine (SVM)}: Demonstrou excelente desempenho em dados de alta dimensionalidade, tornando-se um dos métodos mais utilizados \cite{Lou2024}
    \item \textbf{Random Forest}: Método ensemble que combina múltiplas árvores de decisão, eficaz para grandes datasets \cite{Lou2024}
    \item \textbf{k-Nearest Neighbors (k-NN)}: Método não-paramétrico baseado na proximidade no espaço de características
\end{itemize}

\subsubsection{Terceira Geração: Deep Learning}
A emergência do deep learning revolucionou a classificação de imagens hiperespectrais:
\begin{itemize}
    \item \textbf{Redes Neurais Convolucionais (CNNs)}: Capazes de extrair características espaciais e espectrais automaticamente
    \item \textbf{Redes Neurais Recorrentes (RNNs)}: Eficazes para modelar dependências sequenciais nas bandas espectrais
    \item \textbf{Autoencoders}: Utilizados para redução de dimensionalidade e extração de características não-supervisionada
    \item \textbf{Transformers}: Arquiteturas baseadas em atenção que demonstram excelente desempenho em tarefas de classificação \cite{Lou2024}
\end{itemize}

\section{Tecnologia de Drones para Sensoriamento Hiperespectral}\label{sec:drones_hiperespectral}

\subsection{Vantagens dos Sistemas Hiperespectrais Embarcados em Drones}
A integração de sensores hiperespectrais em plataformas de drones oferece vantagens significativas em comparação com plataformas tradicionais \cite{Shin2024}:

\subsubsection{Flexibilidade Operacional}
Os drones proporcionam alta flexibilidade para aquisição de dados em diferentes condições temporais e espaciais, permitindo monitoramento adaptado às necessidades específicas de cada aplicação.

\subsubsection{Resolução Espacial Superior}
A capacidade de voo em baixas altitudes permite a aquisição de imagens com resolução espacial muito superior à obtida por satélites, chegando a centímetros por pixel.

\subsubsection{Minimização de Interferências Atmosféricas}
O voo em baixas altitudes reduz significativamente os efeitos de dispersão atmosférica e absorção, resultando em dados espectrais de maior qualidade \cite{Shin2024}.

\subsubsection{Custo-Benefício}
Comparado a plataformas aerotransportadas tripuladas, os drones oferecem uma solução muito mais econômica para aquisição de dados hiperespectrais em áreas relativamente pequenas.

\subsection{Desafios Técnicos}
Apesar das vantagens, os sistemas hiperespectrais em drones apresentam desafios únicos \cite{Shin2024}:

\subsubsection{Estabilidade de Plataforma}
As instabilidades inerentes aos drones podem introduzir distorções geométricas que requerem correções específicas durante o pré-processamento.

\subsubsection{Limitações de Payload}
O peso e consumo energético dos sensores hiperespectrais impõem restrições no tempo de voo e na escolha da plataforma aérea.

\subsubsection{Calibração e Correção}
A necessidade de correções radiométricas e geométricas específicas para dados adquiridos por drones requer protocolos de calibração adaptados.

\section{Pré-processamento de Imagens Hiperespectrais de Drones}\label{sec:preprocessamento}

\subsection{Correção Radiométrica}
A correção radiométrica é fundamental para converter os valores digitais capturados pelo sensor em valores de reflectância de superfície, removendo efeitos indesejados de iluminação e resposta do sensor \cite{Shin2024}.

\subsubsection{Método de Linha Empírica (Empirical Line Method - ELM)}
O ELM é uma das técnicas mais utilizadas para correção radiométrica em dados hiperespectrais de drones. O método utiliza alvos de referência com reflectância conhecida, coletados simultaneamente à aquisição das imagens, para estabelecer uma relação linear entre os valores digitais do sensor e a reflectância real \cite{Shin2024}.

Segundo \cite{Shin2024}, a aplicação do ELM demonstrou melhorias de 5-55\% na precisão da reflectância para painéis de referência, resultando em perfis espectrais mais uniformes ao longo das diferentes bandas, com correlações de 0.97-0.99 entre valores medidos e corrigidos.

\subsubsection{Outras Técnicas de Correção}
\begin{itemize}
    \item \textbf{Correção Atmosférica}: Modelagem dos efeitos de dispersão e absorção atmosférica
    \item \textbf{Correção de Sombreamento}: Compensação de variações de iluminação causadas por topografia
    \item \textbf{Calibração de Ganho e Offset}: Ajuste da resposta linear do sensor
\end{itemize}

\subsection{Correção Geométrica}
A correção geométrica é essencial para garantir o alinhamento espacial preciso das imagens e permitir a integração com outros dados geoespaciais \cite{Shin2024}.

\subsubsection{Transformação Rubber Sheet}
A transformação rubber sheet utilizando pontos de controle terrestres tem se mostrado eficaz para correção de distorções causadas por variações na orientação do sensor e trajetória de voo. \cite{Shin2024} reporta erros RMS de 0.00 a 0.081 m na direção leste-oeste e 0.00 a 0.076 m na direção norte-sul, com erro RMS geral de 0.031 metros para 100 pontos, superando padrões industriais de precisão.

\subsubsection{Mosaicagem de Imagens}
O processo de mosaicagem combina múltiplas imagens individuais para criar uma representação abrangente da área de estudo, considerando sobreposições entre imagens e minimizando descontinuidades espectrais \cite{Shin2024}.

\section{Técnicas de Redução de Dimensionalidade}\label{sec:reducao_dimensionalidade}

\subsection{Análise de Componentes Principais (PCA)}
A PCA é uma das técnicas mais utilizadas para redução de dimensionalidade em dados hiperespectrais, transformando o conjunto original de bandas correlacionadas em um novo conjunto de componentes principais não-correlacionados \cite{Lou2024}.

A transformação PCA pode ser expressa matematicamente como:
\begin{equation}
Y = XW
\end{equation}
onde $X$ é a matriz de dados original, $W$ é a matriz de autovetores da matriz de covariância, e $Y$ representa os dados transformados.

\subsection{t-Distributed Stochastic Neighbor Embedding (t-SNE)}
O t-SNE é uma técnica não-linear de redução de dimensionalidade particularmente eficaz para visualização de dados de alta dimensionalidade, preservando a estrutura local dos dados no espaço de menor dimensionalidade.

\subsection{Outras Técnicas}
\begin{itemize}
    \item \textbf{Independent Component Analysis (ICA)}: Separa sinais espectrais misturados em componentes independentes
    \item \textbf{Linear Discriminant Analysis (LDA)}: Maximiza a separabilidade entre classes
    \item \textbf{Minimum Noise Fraction (MNF)}: Ordena bandas por razão sinal-ruído
\end{itemize}

\section{Deep Learning para Classificação Hiperespectral}\label{sec:deep_learning}

\subsection{Redes Neurais Convolucionais (CNNs)}
As CNNs têm demonstrado desempenho superior na classificação de imagens hiperespectrais devido à sua capacidade de extrair características espaciais e espectrais automaticamente \cite{Lou2024}.

\subsubsection{CNNs 2D}
As CNNs 2D processam cada banda espectral separadamente como uma imagem 2D, sendo eficazes para extração de características espaciais. A arquitetura típica inclui:
\begin{itemize}
    \item Camadas convolucionais para extração de características
    \item Camadas de pooling para redução de dimensionalidade
    \item Camadas totalmente conectadas para classificação final
\end{itemize}

\subsubsection{CNNs 3D}
As CNNs 3D processam o cubo hiperespectral como um todo, capturando tanto informações espaciais quanto espectrais simultaneamente. Esta abordagem é particularmente eficaz para dados hiperespectrais devido à natureza tridimensional dos dados.

\subsection{Arquiteturas Transformer}
As arquiteturas Transformer, baseadas em mecanismos de atenção, têm demonstrado excelente desempenho em tarefas de classificação hiperespectral \cite{Lou2024}. O mecanismo de self-attention permite modelar dependências de longo alcance tanto no domínio espacial quanto espectral.

\subsection{Outras Arquiteturas de Deep Learning}
\begin{itemize}
    \item \textbf{Autoencoders}: Para aprendizado não-supervisionado de representações
    \item \textbf{Generative Adversarial Networks (GANs)}: Para aumento de dados e geração sintética
    \item \textbf{Graph Neural Networks (GNNs)}: Para explorar relações topológicas entre pixels
\end{itemize}

\section{Aplicações em Agricultura de Precisão}\label{sec:agricultura_precisao}

\subsection{Monitoramento da Saúde das Culturas}
O sensoriamento hiperespectral permite a detecção precoce de estresses nas culturas através da análise de mudanças sutis nas assinaturas espectrais da vegetação \cite{Shin2024}. Índices espectrais específicos podem indicar:
\begin{itemize}
    \item Deficiências nutricionais (nitrogênio, fósforo, potássio)
    \item Estresse hídrico
    \item Presença de pragas e doenças
    \item Estado fenológico das culturas
\end{itemize}

\subsection{Mapeamento de Produtividade}
A correlação entre características espectrais e produtividade permite a geração de mapas de produtividade preditivos, auxiliando na otimização do manejo agrícola.

\subsection{Aplicação Variável de Insumos}
Os mapas gerados a partir da classificação LULC hiperespectral podem orientar a aplicação variável de fertilizantes, pesticidas e irrigação, otimizando o uso de recursos e reduzindo impactos ambientais.

\section{Trabalhos Relacionados}\label{sec:trabalhos_relacionados}

\subsection{Estudos de Revisão}
\cite{Lou2024} conduziu uma revisão abrangente sobre classificação LULC usando imagens hiperespectrais, identificando quatro abordagens principais: revisão sistemática da classificação LULC hiperespectral, compilação e análise de datasets específicos, exploração de métodos tradicionais e de deep learning, e análise de trajetórias futuras de desenvolvimento.

\subsection{Implementações com Drones}
\cite{Shin2024} apresentou um estudo focado em métodos robustos de correção radiométrica e geométrica para imageamento hiperespectral baseado em drones em aplicações agrícolas, demonstrando a eficácia do ELM e transformação rubber sheet para preprocessamento de dados.

\subsection{Sistemas Embarcados}
\cite{Lim2022} investigou a viabilidade de sistemas embarcados em tempo real para imageamento hiperespectral compressivo, analisando requisitos computacionais, uso de memória e largura de banda, e propondo otimizações para alcançar processamento em tempo real.

\subsection{Lacunas Identificadas na Literatura}
A revisão da literatura identifica várias lacunas importantes:

\begin{enumerate}
    \item \textbf{Padronização de Metodologias}: Falta de protocolos padronizados para aquisição e processamento de dados hiperespectrais de drones
    \item \textbf{Datasets Públicos}: Escassez de datasets públicos anotados específicos para aplicações LULC com drones hiperespectrais
    \item \textbf{Validação em Condições Reais}: Necessidade de mais estudos validando metodologias em diferentes condições ambientais e tipos de culturas
    \item \textbf{Integração de Técnicas}: Falta de frameworks integrados que combinem eficientemente preprocessamento, redução de dimensionalidade e classificação
    \item \textbf{Aplicabilidade Prática}: Limitado número de estudos focando na implementação prática e transferência de tecnologia para o setor produtivo
\end{enumerate}

\section{Síntese do Levantamento}\label{sec:sintese}

Esta revisão da literatura estabelece o contexto científico e tecnológico para o desenvolvimento desta dissertação. Os principais pontos identificados são:

\subsection{Estado da Arte}
\begin{itemize}
    \item O sensoriamento hiperespectral com drones representa uma tecnologia emergente com grande potencial para aplicações em agricultura de precisão
    \item Técnicas de deep learning demonstram desempenho superior aos métodos tradicionais para classificação LULC hiperespectral
    \item Métodos de correção radiométrica e geométrica específicos para drones são essenciais para garantir a qualidade dos dados
\end{itemize}

\subsection{Desafios Técnicos}
\begin{itemize}
    \item Processamento eficiente de dados de alta dimensionalidade
    \item Desenvolvimento de pipelines integrados de processamento
    \item Validação em condições operacionais reais
    \item Transferência de tecnologia para aplicações práticas
\end{itemize}

\subsection{Oportunidades de Pesquisa}
Esta dissertação visa contribuir para o preenchimento das lacunas identificadas através de:
\begin{itemize}
    \item Desenvolvimento de uma metodologia integrada para classificação LULC com drones hiperespectrais
    \item Implementação e validação de técnicas avançadas de deep learning
    \item Avaliação sistemática em diferentes condições experimentais
    \item Estabelecimento de protocolos para aplicação prática da tecnologia
\end{itemize}

Os capítulos subsequentes detalham a metodologia desenvolvida e os resultados obtidos, contribuindo para o avanço do conhecimento nesta área de pesquisa emergente e estratégica.

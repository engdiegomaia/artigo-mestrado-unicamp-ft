% Capítulo de Levantamento Bibliográfico / Trabalhos Relacionados
\chapter{Levantamento Bibliográfico}\label{chp:levantamento}

% TODO: Expandir cada seção com revisão detalhada da literatura
% Introdução ao capítulo
Este capítulo apresenta uma revisão sistemática da literatura sobre processamento de imagens hiperespectrais, arquiteturas de processamento paralelo (FPGA e GPU) e trabalhos relacionados que comparam diferentes plataformas computacionais.

\section{Processamento de Imagens Hiperespectrais}\label{sec:proc_hiperespectral}

% TODO: Definir conceitos fundamentais
\subsection{Conceitos Fundamentais}
% - Definição de imagem hiperespectral
% - Diferenças entre multiespectral e hiperespectral  
% - Características dos dados hiperespectrais (volume, dimensionalidade, ruído)

\subsection{Aplicações}
% TODO: Expandir com exemplos concretos e citações
As imagens hiperespectrais encontram aplicações em diversas áreas:
\begin{itemize}
    \item \textbf{Agricultura de Precisão}: Monitoramento de culturas, detecção de pragas e doenças
    \item \textbf{Mineração}: Identificação de minerais e mapeamento geológico
    \item \textbf{Medicina}: Diagnóstico por imagem e cirurgia guiada
    \item \textbf{Monitoramento Ambiental}: Qualidade da água, desmatamento, poluição
    \item \textbf{Defesa e Segurança}: Detecção de alvos e camuflagem
\end{itemize}

% TODO: Adicionar estatísticas de crescimento do mercado
% TODO: Citar principais sensores e missões espaciais

\subsection{Desafios Computacionais}
% TODO: Quantificar volumes de dados e complexidade computacional
% - Volume de dados (GB/TB por imagem)
% - Complexidade temporal dos algoritmos
% - Requisitos de memória
% - Necessidade de processamento em tempo real

\section{Algoritmos de Processamento}\label{sec:algoritmos}

\subsection{Pré-processamento}
% TODO: Detalhar cada etapa com equações e referências
\begin{itemize}
    \item \textbf{Correção Radiométrica}: Compensação de variações de iluminação
    \item \textbf{Correção Atmosférica}: Remoção de efeitos atmosféricos
    \item \textbf{Redução de Ruído}: Filtragem espacial e espectral
    \item \textbf{Normalização}: Padronização dos valores espectrais
\end{itemize}

\subsection{Redução de Dimensionalidade}
% TODO: Expandir com formulações matemáticas
\subsubsection{Análise de Componentes Principais (PCA)}
% - Formulação matemática
% - Complexidade computacional O(n³)
% - Vantagens e limitações
% - Variações (Kernel PCA, Incremental PCA)

\subsubsection{Outras Técnicas}
% - Independent Component Analysis (ICA)
% - Linear Discriminant Analysis (LDA)
% - Minimum Noise Fraction (MNF)

\subsection{Classificação}
% TODO: Detalhar algoritmos com complexidade computacional
\subsubsection{Support Vector Machine (SVM)}
% - Formulação do problema de otimização
% - Kernels (linear, RBF, polinomial)
% - Complexidade de treinamento e predição

\subsubsection{Outros Classificadores}
% - Random Forest
% - Redes Neurais Convolucionais
% - k-Nearest Neighbors (k-NN)

\section{Arquiteturas de Processamento Paralelo}\label{sec:arquiteturas}

\subsection{Field-Programmable Gate Arrays (FPGA)}\label{subsec:fpga}
% TODO: Expandir com detalhes técnicos e citações
\subsubsection{Características Arquiteturais}
% - Estrutura de CLBs, LUTs, flip-flops
% - Interconexões programáveis
% - Blocos de memória e DSP
% - Consumo energético

\subsubsection{Vantagens para Processamento Hiperespectral}
\begin{itemize}
    \item Paralelismo massivo configurável
    \item Baixo consumo energético
    \item Processamento em pipeline
    \item Determinismo temporal
    \item Customização da arquitetura
\end{itemize}

\subsubsection{Limitações}
% - Complexidade de desenvolvimento
% - Ferramentas de síntese
% - Recursos finitos
% - Frequência de operação limitada

\subsection{Graphics Processing Units (GPU)}\label{subsec:gpu}
% TODO: Expandir com arquitetura CUDA/OpenCL
\subsubsection{Arquitetura CUDA}
% - Streaming Multiprocessors (SM)
% - Hierarquia de memória
% - Modelo de programação (threads, warps, blocos)
% - Ocupação e latência

\subsubsection{Vantagens para Processamento Hiperespectral}
\begin{itemize}
    \item Alto throughput para operações paralelas
    \item Ferramentas de desenvolvimento maduras
    \item Bibliotecas otimizadas (cuBLAS, cuFFT)
    \item Suporte a ponto flutuante de alta precisão
    \item Escalabilidade com múltiplas GPUs
\end{itemize}

\subsubsection{Limitações}
% - Alto consumo energético
% - Latência de transferência de dados
% - Modelo de programação específico
% - Dependência de drivers

\section{Trabalhos Relacionados}\label{sec:trabalhos_relacionados}

% TODO: Organizar por categoria de contribuição
\subsection{Implementações FPGA para Processamento Hiperespectral}
% TODO: Revisar literatura específica com análise crítica
% - Trabalhos pioneiros
% - Algoritmos implementados
% - Métricas de desempenho reportadas
% - Limitações identificadas

\subsection{Implementações GPU para Processamento Hiperespectral}
% TODO: Revisar implementações CUDA/OpenCL
% - Primeiras implementações
% - Otimizações propostas
% - Benchmarks realizados
% - Comparações com CPU

\subsection{Estudos Comparativos}
% TODO: Identificar lacunas na literatura
\subsubsection{FPGA vs CPU}
% - Métricas comparadas
% - Cenários avaliados
% - Conclusões principais

\subsubsection{GPU vs CPU}
% - Speedups reportados
% - Eficiência energética
% - Casos de uso

\subsubsection{FPGA vs GPU}
% TODO: Identificar poucos trabalhos existentes - justificar contribuição
% - Trabalhos escassos na literatura
% - Limitações dos estudos existentes
% - Necessidade de comparação sistemática

\section{Síntese do Levantamento}\label{sec:sintese}

% TODO: Resumir lacunas identificadas
Este levantamento identifica as seguintes lacunas na literatura:
\begin{enumerate}
    \item Ausência de comparações sistemáticas FPGA vs GPU para processamento hiperespectral completo
    \item Falta de métricas padronizadas para avaliação de diferentes arquiteturas
    \item Implementações não reproduzíveis ou com datasets proprietários
    \item Análises limitadas de trade-offs energéticos
    \item Diretrizes insuficientes para seleção de plataforma
\end{enumerate}

% TODO: Posicionar a contribuição deste trabalho
Esta dissertação visa preencher essas lacunas através de uma metodologia sistemática e implementações reproduzíveis, conforme detalhado no \Capitulo{chp:metodologia}.

% TODO: Adicionar tabela resumo dos trabalhos relacionados
% TODO: Incluir timeline da evolução das técnicas
